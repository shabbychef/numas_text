%%%%(c) COPYRIGHT NOTICE%FOLDUP
%%%%(c)
%%%%(c)  This file is a portion of the source for the textbook
%%%%(c)
%%%%(c)    Numerical Methods Course Notes,
%%%%(c)    Copyright 2004-2010 by Steven E. Pav
%%%%(c)
%%%%(c)  See the file COPYING.txt for copying conditions
%%%%(c)
%%%%(c)%UNFOLD

%%throat clearing%FOLDUP
\typeout{-- octave_backslash_lsqr.tex}
\typeout{-- N� 2004-2010 Steven E. Pav}
%UNFOLD

%%local commands%FOLDUP
%UNFOLD

%%%%%%%%%%%%%%%%%%%%%%%%%%%%%%%%%%%%%%%%%%%%%%%
\subsection{Ordinary Least Squares in \octmat}%FOLDUP

The ordinary least squares best approximant is so fundamental that it is
hard-wired into \octmat's system solve operator.  The general equation 
we wish to solve is 
\[\Mtx{A}\vect{c} = \vect{y}.\]
This system is overdetermined: there are more rows than columns in \Mtx{A},
\ie more equations than unknown variables.  However, in \octmat, this is
solved just like the case where \Mtx{A} is square:

	\begin{verbatim}
	octave:1> c = A \ y;
	\end{verbatim}
Thus, unless specifically told otherwise, you should not implement the
ordinary least squares solution method in \octmat.  Under the hood,
\octmat is \emph{not} using the normal equations method, and does not
suffer the increased condition number of the normal equations method.
%UNFOLD

%for vim modeline: (do not edit)
% vim:ts=2:sw=2:tw=79:fdm=marker:fmr=FOLDUP,UNFOLD:cms=%%s:tags=tags;:syntax=tex:filetype=tex:ai:si:cin:nu:fo=croqt:cino=p0t0c5(0:
