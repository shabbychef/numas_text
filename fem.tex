%%%%(c) COPYRIGHT NOTICE%FOLDUP
%%%%(c)
%%%%(c)  This file is a portion of the source for the textbook
%%%%(c)
%%%%(c)    Numerical Methods Course Notes,
%%%%(c)    Copyright 2004-2010 by Steven E. Pav
%%%%(c)
%%%%(c)  See the file COPYING.txt for copying conditions
%%%%(c)
%%%%(c)%UNFOLD

%%throat clearing%FOLDUP
\typeout{-- fem.tex}
\typeout{-- N� 2004-2010 Steven E. Pav}
%UNFOLD

%%local commands%FOLDUP
%UNFOLD

%THIS CHAPTER IS ONLY A STUB!
\typeout{-- fem.tex IS ONLY A STUB!}

\chapter{Partial Differential Equations by Finite Element Method}\label{chap:fem}

Ordinary Differential Equations, or ODEs, are used in approximating some
physical systems.  Some classical uses include simulation of the growth of
populations and trajectory of a particle.  They are usually easier to solve or
approximate than the more general Partial Differential Equations, PDEs, which
can contain partial derivatives.

Much of the background material of this chapter should be familiar to the
student of calculus.  We will focus more on the approximation of solutions,
than on analytical solutions.  For more background on ODEs, see any of the
standard texts, \eg \cite{Sjcalc5}.

....

%%%%%%%%%%%%%%%%%%%%%%%%%%%%%%%%%%%%%%%%%%%%%%%
\section{etc}%FOLDUP
%UNFOLD
%%%%%%%%%%%%%%%%%%%%%%%%%%%%%%%%%%%%%%%%%%%%%%%
%\section{Exercises}%FOLDUP
\begin{bkexs}
%%%%%%%%%%%%%%%%%%%%%%%%%%%%%%%%
\item an exercise.


%%%%%%%%%%%%%%%%%%%%%%%%%%%%%%%%
\end{bkexs}
%UNFOLD

%for vim modeline: (do not edit)
% vim:ts=2:sw=2:tw=79:fdm=marker:fmr=FOLDUP,UNFOLD:cms=%%s:tags=tags;:syntax=tex:filetype=tex:ai:si:cin:nu:fo=croqt:cino=p0t0c5(0:
