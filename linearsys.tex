%%%%(c) COPYRIGHT NOTICE%FOLDUP
%%%%(c)
%%%%(c)  This file is a portion of the source for the textbook
%%%%(c)
%%%%(c)    Numerical Methods Course Notes,
%%%%(c)    Copyright 2004-2010 by Steven E. Pav
%%%%(c)
%%%%(c)  See the file COPYING.txt for copying conditions
%%%%(c)
%%%%(c)%UNFOLD


%%throat clearing%FOLDUP
\typeout{-- linearsys.tex}
\typeout{-- N� 2004-2010 Steven E. Pav}
%UNFOLD

%%local commands%FOLDUP
\providecommand{\avecUL}[3]{\ensuremath{\neUL{\vect{#1}}{#2}{#3}}\xspace}
\providecommand{\thex}[1]{\avecUL{x}{\wrapNeParens{#1}}{}}
\providecommand{\they}[1]{\avecUL{y}{\wrapNeParens{#1}}{}}
\providecommand{\thez}[1]{\avecUL{z}{\wrapNeParens{#1}}{}}
\providecommand{\thexi}[2]{\avecUL{x}{\wrapNeParens{#1}}{#2}}
\providecommand{\thee}[1]{\avecUL{e}{\wrapNeParens{#1}}{}}
\providecommand{\thexc}{\avecUL{x}{*}{}}
\providecommand{\ther}[1]{\avecUL{r}{\wrapNeParens{#1}}{}}

\providecommand{\UU}[2]{\ensuremath{U^{#2}_{#1}}\xspace}

\providecommand{\dee}[1]{\ensuremath{\mathcal{D}_{#1}}\xspace}
\providecommand{\pee}[1]{\ensuremath{\mathcal{P}_{#1}}\xspace}
%UNFOLD

\chapter{Solving Linear Systems}
\label{chap:linearsys}

A number of problems in numerical analysis can be reduced to, or approximated
by, a system of linear equations. 

%%%%%%%%%%%%%%%%%%%%%%%%%%%%%%%%%%%%%%%%%%%%%%%
\section{Gaussian Elimination with \Naive Pivoting}%FOLDUP
\label{sec:gaussianE}

\index{Gaussian Elimination, \Naive}
Our goal is the automatic solution of systems of linear equations:
\[
\begin{array}{cccccccccccc}
a_{11} x_1 & + & a_{12} x_2 & + & a_{13} x_3 & + & \cdots & + & a_{1n} x_n & = & b_1\\
a_{21} x_1 & + & a_{22} x_2 & + & a_{23} x_3 & + & \cdots & + & a_{2n} x_n & = & b_2\\
a_{31} x_1 & + & a_{32} x_2 & + & a_{33} x_3 & + & \cdots & + & a_{3n} x_n & = & b_3\\
\vdots & & \vdots & & \vdots & & \ddots & & \vdots & & \vdots \\
a_{n1} x_1 & + & a_{n2} x_2 & + & a_{n3} x_3 & + & \cdots & + & a_{nn} x_n & = & b_n
\end{array}
\]

In these equations, the $a_{ij}$ and $b_i$ are given real numbers.  We also
write this as
\[\Mtx{A} \vect{x} = \vect{b},\]
where \Mtx{A} is a matrix, whose element in the \kth{i} row and \kth{j} column
is $a_{ij},$ and \vect{b} is a column vector, whose \kth{i} entry is $b_i.$

This gives the easier way of writing this equation:
\begin{eqnarray}
\Bracks{\begin{array}{ccccc}
a_{11} & a_{12} & a_{13} & \cdots & a_{1n}\\
a_{21} & a_{22} & a_{23} & \cdots & a_{2n}\\
a_{31} & a_{32} & a_{33} & \cdots & a_{3n}\\
\vdots & \vdots & \vdots & \ddots & \vdots\\
a_{n1} & a_{n2} & a_{n3} & \cdots & a_{nn}
\end{array}}
\Bracks{\begin{array}{c} x_1 \\ x_2 \\ x_3 \\ \vdots \\ x_n\end{array}}
= \Bracks{\begin{array}{c} b_1 \\ b_2 \\ b_3 \\ \vdots \\ b_n\end{array}}
\label{eqn:thelinearsys}
\end{eqnarray}


%
%%%%%%%%%%%%%%%%%%%%%%%%%%%%%%%%%%%%%%%%%%%%%%%
\subsection{Elementary Row Operations}%FOLDUP

You may remember that one way to solve linear equations is by applying
\emph{elementary row operations} \index{elementary row operations} 
to a given equation of the system.  For
example, if we are trying to solve the given system of equations, they should
have the same solution as the following system:

\[
\Bracks{\begin{array}{ccccc}
a_{11} & a_{12} & a_{13} & \cdots & a_{1n}\\
a_{21} & a_{22} & a_{23} & \cdots & a_{2n}\\
a_{31} & a_{32} & a_{33} & \cdots & a_{3n}\\
\vdots & \vdots & \vdots & \ddots & \vdots\\
\kappa a_{i1} & \kappa a_{i2} & \kappa a_{i3} & \cdots & \kappa a_{in}\\
\vdots & \vdots & \vdots & \ddots & \vdots\\
a_{n1} & a_{n2} & a_{n3} & \cdots & a_{nn}
\end{array}}
\Bracks{\begin{array}{c} x_1 \\ x_2 \\ x_3 \\ \vdots \\ x_i \\ \vdots \\ x_n\end{array}}
= \Bracks{\begin{array}{c} b_1 \\ b_2 \\ b_3 \\ \vdots \\ \kappa b_i \\ \vdots \\ b_n\end{array}}
\]

where $\kappa$ is some given number which is \emph{not zero}.
It suffices to solve this system of linear equations, as it has the same
solution(s) as our original system.  Multiplying a row of the system by a
nonzero constant is one of the elementary row operations.  

The second elementary row operation is to replace a row by the sum of that row
and a constant times another.  Thus, for example, the following system of equations
has the same solution as the original system:


\[
\Bracks{\begin{array}{ccccc}
a_{11} & a_{12} & a_{13} & \cdots & a_{1n}\\
a_{21} & a_{22} & a_{23} & \cdots & a_{2n}\\
a_{31} & a_{32} & a_{33} & \cdots & a_{3n}\\
\vdots & \vdots & \vdots & \ddots & \vdots\\
a_{(i-1)1} & a_{(i-1)2} & a_{(i-1)3} & \cdots & a_{(i-1)n}\\
a_{i1} + \beta a_{j1} & a_{i2} + \beta a_{j2} & a_{i3} + \beta a_{j3} & \cdots
& a_{in} + \beta a_{jn}\\
\vdots & \vdots & \vdots & \ddots & \vdots\\
a_{n1} & a_{n2} & a_{n3} & \cdots & a_{nn}
\end{array}}
\Bracks{\begin{array}{c} x_1 \\ x_2 \\ x_3 \\ \vdots \\ x_{(i-1)} \\ x_i \\ \vdots \\ x_n\end{array}}
= \Bracks{\begin{array}{c} b_1 \\ b_2 \\ b_3 \\ \vdots \\ b_{(i-1)} \\ b_i +
\beta b_j \\ \vdots \\ b_n\end{array}}
\]

We have replaced the \kth{i} row by the \kth{i} row plus $\beta$ times the
\kth{j} row.

The third elementary row operation is to switch rows:
\[
\Bracks{\begin{array}{ccccc}
a_{11} & a_{12} & a_{13} & \cdots & a_{1n}\\
a_{31} & a_{32} & a_{33} & \cdots & a_{3n}\\
a_{21} & a_{22} & a_{23} & \cdots & a_{2n}\\
\vdots & \vdots & \vdots & \ddots & \vdots\\
a_{n1} & a_{n2} & a_{n3} & \cdots & a_{nn}
\end{array}}
\Bracks{\begin{array}{c} x_1 \\ x_2 \\ x_3 \\ \vdots \\ x_n\end{array}}
= \Bracks{\begin{array}{c} b_1 \\ b_3 \\ b_2 \\ \vdots \\ b_n\end{array}}
\]
We have here switched the second and third rows.  The purpose of this e.r.o. is
mainly to make things look nice.  

Note that none of the e.r.o.'s change the structure of the solution vector
\vect{x}.  For this reason, it is customary to drop the solution vector
entirely and to write the matrix \Mtx{A} and the vector \vect{b} together in
\emph{augmented form}:
\[
\Parens{\begin{array}{ccccc|c}
a_{11} & a_{12} & a_{13} & \cdots & a_{1n} & b_1\\
a_{21} & a_{22} & a_{23} & \cdots & a_{2n} & b_2\\
a_{31} & a_{32} & a_{33} & \cdots & a_{3n} & b_3\\
\vdots & \vdots & \vdots & \ddots & \vdots &\\
a_{n1} & a_{n2} & a_{n3} & \cdots & a_{nn} & b_n
\end{array}}
\]

The idea of Gaussian Elimination is to use the elementary row operations to
put a system into \emph{upper triangular form} then use \emph{back
substitution}.  We'll give an example here:

\begin{bkexprob}%stolen from mcQuarrie p 412, then altered enough?%FOLDUP
Solve the set of linear equations:
\begin{align*}
x_1 + x_2 - x_3 &= 2\\
-3 x_1 - 4 x_2 + 4 x_3 &= -7\\
2 x_1 + 1 x_2 + 1 x_3 &= 7
\end{align*}
\begin{bksolution}
We start by rewriting in the augmented form:
\[
\Parens{\begin{array}{rrr|r}
1 & 1 & -1 & 2\\
-3 & -4 & 4 & -7\\
2 & 1 & 1 & 7
\end{array}}
\]
We add $3$ times the first row to the second, and $-2$ times the first row to
the third to get:
\[
\Parens{\begin{array}{rrr|r}
1 & 1 & -1 & 2\\
0 & -1 & 1 & -1\\
0 & -1 & 3 & 3 
\end{array}}
\]
We now add $-1$ times the second row to the third row to get:
\[
\Parens{\begin{array}{rrr|r}
1 & 1 & -1 & 2\\
0 & -1 & 1 & -1 \\
0 & 0 & 2 & 4
\end{array}}
\]
The matrix is now in upper triangular form: there are no nonzero entries below
the diagonal.  This corresponds to the set of equations:
\begin{align*}
x_1 + x_2 - x_3 &= 2\\
- x_2 + x_3 &= -1\\
2 x_3 &= 4
\end{align*}
We now solve this by back substitution.  Because the matrix is in upper
triangular form, we can solve $x_3$ by looking only at the last equation;
namely $x_3 = 2.$  However, once $x_3$ is known, the second equation involves
only one unknown, $x_2,$ and can be solved only by $x_2 = 3.$  Then the first
equation has only one unknown, and is solved by $x_1 = 1.$
\end{bksolution}
\end{bkexprob}%UNFOLD

All sorts of funny things can happen when you attempt Gaussian Elimination: it
may turn out that your system has no solution, or has a single solution (as
above), or an infinite number of solutions.  We should expect that an algorithm
for automatic solution of systems of equations should detect these problems.

%UNFOLD
%
%%%%%%%%%%%%%%%%%%%%%%%%%%%%%%%%%%%%%%%%%%%%%%%
\subsection{Algorithm Terminology}%FOLDUP

The method outlined above is fine for solving small systems.  We should like to
devise an algorithm for doing the same thing which can be applied to large
systems of equations.  The algorithm will take the system (in augmented form):
\[
\Parens{\begin{array}{ccccc|c}
a_{11} & a_{12} & a_{13} & \cdots & a_{1n} & b_1\\
a_{21} & a_{22} & a_{23} & \cdots & a_{2n} & b_2\\
a_{31} & a_{32} & a_{33} & \cdots & a_{3n} & b_3\\
\vdots & \vdots & \vdots & \ddots & \vdots &\\
a_{n1} & a_{n2} & a_{n3} & \cdots & a_{nn} & b_n
\end{array}}
\]
The algorithm then selects the first row as the \emph{pivot equation} or
\emph{pivot row}, and the
first element of the first row, $a_{11}$ is the \emph{pivot element}.  The
algorithm then \emph{pivots} on the pivot element to get the system:
\[
\Parens{\begin{array}{ccccc|c}
a_{11} & a_{12} & a_{13} & \cdots & a_{1n} & b_1\\
0 & a'_{22} & a'_{23} & \cdots & a'_{2n} & b'_2\\
0 & a'_{32} & a'_{33} & \cdots & a'_{3n} & b'_3\\
\vdots & \vdots & \vdots & \ddots & \vdots &\\
0 & a'_{n2} & a'_{n3} & \cdots & a'_{nn} & b'_n
\end{array}}
\]
Where
\[
\left.\begin{array}{ccc}
a'_{ij} &= a_{ij} - \Parens{\frac{a_{i1}}{a_{11}}} a_{1j}\\
b'_{i} &= b_{i} - \Parens{\frac{a_{i1}}{a_{11}}} b_1
\end{array} \right\} \qquad \Parens{2 \le i \le n, \, 1 \le j \le n}
\]
Effectively we are carrying out the e.r.o. of replacing the \kth{i} row by the
\kth{i} row minus $\Parens{\frac{a_{i1}}{a_{11}}}$ times the first row.  The
quantity $\Parens{\frac{a_{i1}}{a_{11}}}$ is the \emph{multiplier} \index{multiplier} 
for the \kth{i} row.

Hereafter the algorithm will not alter the first row or first column of the
system.  Thus, the algorithm could be written recursively.  By pivoting on the
second row, the algorithm then generates the system:
\[
\Parens{\begin{array}{ccccc|c}
a_{11} & a_{12} & a_{13} & \cdots & a_{1n} & b_1\\
0 & a'_{22} & a'_{23} & \cdots & a'_{2n} & b'_2\\
0 & 0 & a''_{33} & \cdots & a''_{3n} & b''_3\\
\vdots & \vdots & \vdots & \ddots & \vdots &\\
0 & 0 & a''_{n3} & \cdots & a''_{nn} & b''_n
\end{array}}
\]
In this case
\[
\left.\begin{array}{ccc}
a''_{ij} &=& a'_{ij} - \Parens{\frac{a'_{i2}}{a'_{22}}} a'_{2j}\\
b''_{i} &=& b'_{i} - \Parens{\frac{a'_{i2}}{a'_{22}}} b'_2
\end{array} \right\} \qquad \Parens{3 \le i \le n, \, 1 \le j \le n}
\]
%UNFOLD
%
%%%%%%%%%%%%%%%%%%%%%%%%%%%%%%%%%%%%%%%%%%%%%%%
\subsection{Algorithm Problems}%FOLDUP

The pivoting strategy we examined in this section is called `\naive' because a
real algorithm is a bit more complicated.  The algorithm we have outlined is
far too rigid--it always chooses to pivot on the \kth{k} row during the \kth{k}
step.  This would be bad if the pivot element were zero; in this case all the
multipliers $\frac{a_{ik}}{a_{kk}}$ are not defined.

Bad things can happen if $a_{kk}$ is merely small instead of zero.  Consider
the following example:

\begin{bkexample}%idea stolen from conte and deboor%FOLDUP
Solve the system of equations given by the augmented form:
\[
\Parens{\begin{array}{rr|r}
-0.0590 & 0.2372 & -0.3528\\
0.1080 & -0.4348 & 0.6452
\end{array}}
\]
Note that the exact solution of this system is $x_1 = 10,\,x_2 = 1.$
Suppose, however, that the algorithm uses only 4 significant figures for its 
calculations.  The algorithm, \naively, pivots on the first equation.  The multiplier for the second row
is
\[
\frac{0.1080}{-0.0590} \approx -1.830508...,
\]
which will be rounded to $-1.831$ by the algorithm.

The second entry in the matrix is replaced by
\[
-0.4348 - (-1.831)(0.2372) = -0.4348 + 0.4343 = -0.0005,
\]
where the arithmetic is rounded to four significant figures each time.
There is some serious subtractive cancellation going on here. We have lost
three figures with this subtraction.  The errors get worse from here.
Similarly, the second vector entry becomes:
\[
0.6452 - (-1.831)(-0.3528) = 0.6452 - 0.6460 = -0.0008,
\]
where, again, intermediate steps are rounded to four significant figures, and
again there is subtractive cancelling.
This puts the system in the form
\[
\Parens{\begin{array}{rr|r}
-0.0590 & 0.2372 & -0.3528\\
0 & -0.0005 & -0.0008
\end{array}}
\]
When the algorithm attempts back substitution, it gets the value 
\[x_2 = \frac{-0.0008}{-0.0005} = 1.6.\]
This is a bit off from the actual value of $1$.  The algorithm now finds
\[x_1 = 
%\Parens{1.569 - 1.566 x_2}/0.0003 = 
\Parens{-0.3528 - 0.2372 \cdot 1.6} / -0.059 =
\Parens{-0.3528 - 0.3795}/ -0.059 = 
\Parens{-0.7323}/ - 0.059 = 12.41,\]
where each step has rounding to four significant figures. This is also a bit
off.

\end{bkexample}%UNFOLD
%\begin{bkexample}%stolen from conte and deboor%FOLDUP
%Solve the system of equations given by the augmented form:
%\[
%\Parens{\begin{array}{rr|c}
%0.0003 & 1.566 & 1.569\\
%0.3454 & -2.436 & 1.018
%\end{array}}
%\]
%Note that the exact solution of this system is $x_1 = 10,\,x_2 = 1.$
%Suppose, however, that the algorithm uses only 4 significant figures for its 
%calculations.  The algorithm, \naively, pivots on the first equation.  The multiplier for the second row
%is
%\[
%\frac{0.3454}{0.0003} = 1151.3\bar{3},
%\]
%which will be rounded to $1151$ by the algorithm.
%
%The second entry in the matrix is replaced by
%\[
%-2.436 - (1151)(1.566) = -2.436 - 1802 = -1804,
%\]
%where the arithmetic is rounded to four significant figures each time.
%Similarly, the second vector entry becomes:
%\[
%1.018 - (1151)(1.569) = 1.018 - 1806 = -1805,
%\]
%so the system is in the form
%\[
%\Parens{\begin{array}{rr|c}
%0.0003 & 1.566 & 1.569\\
%0 & -1804 & -1805
%\end{array}}
%\]
%When the algorithm attempts back substitution, the value 
%\[x_2 = \frac{-1805}{-1804} \approx 1.0000554324\]
%is rounded to $1.001$.  This is not so far off. However, it
%now computes
%\[x_1 = 
%%\Parens{1.569 - 1.566 x_2}/0.0003 = 
%\Parens{1.569 - 1.566 \cdot 1.001}/0.0003 = 
%\Parens{1.569 - 1.568}/0.0003 = 
%\Parens{0.001}/0.0003 = 3.3\bar{3},\]
%which the computer rounds to $3.333$.  This is far from the right answer.
%Looks like there was some subtractive cancelling in the last part which made
%things bad.
%
%\end{bkexample}%UNFOLD
%UNFOLD
%UNFOLD
%%%%%%%%%%%%%%%%%%%%%%%%%%%%%%%%%%%%%%%%%%%%%%%
\section{Pivoting Strategies for Gaussian Elimination}%FOLDUP
\label{sec:gaussianP}
\depson{sec:gaussianE}{sec:gaussianP}

Gaussian Elimination can fail when performed in the wrong order.  If the
algorithm selects a zero pivot, the multipliers are undefined, which is no
good.  We also saw that a pivot small in magnitude can cause failure. As here:
\begin{align*}
\epsilon x_1 + x_2 &= 1\\
x_1 + x_2 &= 2
\end{align*}

The \naive algorithm solves this as
\begin{align*}
x_2 & = \frac{2-\oneby{\epsilon}}{1-\oneby{\epsilon}} = 1 - \frac{\epsilon}{1-\epsilon}\\
x_1 & = \frac{1-x_2}{\epsilon} = \oneby{1-\epsilon}
\end{align*}
If $\epsilon$ is very small, then \oneby{\epsilon} is enormous compared to both
$1$ and $2.$  With poor rounding, the algorithm solves $x_2$ as $1.$  Then it
solves $x_1 = 0.$  This is nearly correct for $x_2,$ but is an awful
approximation for $x_1.$  Note that this choice of $x_1,x_2$ satisfies the
first equation, but not the second.

Now suppose the algorithm changed
the order of the equations, then solved:
\begin{align*}
x_1 + x_2 &= 2\\
\epsilon x_1 + x_2 &= 1
\end{align*}
The algorithm solves this as
\begin{align*}
x_2 & = \frac{1-2\epsilon}{1 - \epsilon}\\
x_1 & = 2-x_2
\end{align*}
There's no problem with rounding here.

The problem is not the small entry \emph{per se}:  Suppose we use an e.r.o. to
scale the first equation, then use \naive G.E.:
\begin{align*}
x_1 + \oneby{\epsilon} x_2 &= \oneby{\epsilon}\\
x_1 + x_2 &= 2
\end{align*}
This is still solved as 
\begin{align*}
x_2 & = \frac{2-\oneby{\epsilon}}{1-\oneby{\epsilon}}\\
x_1 & = \frac{1-x_2}{\epsilon},
\end{align*}
and rounding is still a problem.

%%%%%%%%%%%%%%%%%%%%%%%%%%%%%%%%%%%%%%%%%%%%%%%
\subsection{Scaled Partial Pivoting}%FOLDUP

The \naive G.E. algorithm uses the rows \onetox{n-1} in order as pivot
equations.  As shown above, this can cause errors.  Better is to pivot first on
row $\ell_1,$ then row $\ell_2,$ etc, until finally pivoting on row
$\ell_{n-1},$ for some permutation \setBIdx{\ell_i}{i=1}{n} of the integers
\onetox{n}.  The strategy of \emph{scaled partial pivoting} is to compute this
permutation so that G.E. works well.  

In light of our example, we want to pivot on an element which is not small
compared to other elements in its row.  So our algorithm first determines
``smallness'' by calculating a scale, row-wise:
\[
s_i = \max_{1\le j \le n} \abs{a_{ij}}.
\]
The scales are only computed once.

Then the first pivot, $\ell_1$, is chosen to be the $i$ such that 
\[
\frac{\abs{a_{i,1}}}{s_i}
\]
is maximized.  The algorithm pivots on row $\ell_1,$ producing a bunch of zeros
in the first column.  Note that the algorithm should \emph{not} rearrange the
matrix--this takes too much work.

The second pivot, $\ell_2$, is chosen to be the $i$ such that 
\[
\frac{\abs{a_{i,2}}}{s_i}
\]
is maximized, but without choosing $\ell_2 = \ell_1.$  The algorithm pivots on row $\ell_2,$ producing a bunch of zeros
in the second column.  

In the \kth{k} step $\ell_k$ is chosen to be the $i$ not among
$\ell_1,\ell_2,\ldots,\ell_{k-1}$ such that 
\[
\frac{\abs{a_{i,k}}}{s_i}
\]
is maximized.  The algorithm pivots on row $\ell_k,$ producing a bunch of zeros
in the \kth{k} column.  

The slick way to implement this is to first set $\ell_i = i$ for
$i=\onetox{n}.$ Then rearrange this vector in a kind of ``bubble sort'': when
you find the index that should be $\ell_1,$ swap them, \ie find the $j$ such
that $\ell_j$ should be the first pivot and switch the values of
$\ell_1,\ell_j.$

Then at the \kth{k} step, search only those indices in the tail of this vector:
\ie only among $\ell_j$ for $k \le j \le n,$ and perform a swap.

%UNFOLD
%%%%%%%%%%%%%%%%%%%%%%%%%%%%%%%%%%%%%%%%%%%%%%%
\subsection{An Example}%FOLDUP

We present an example of using {scaled partial pivoting} with G.E.
It's hard to come up with an example where the numbers do not come out as ugly
fractions.  We'll look at a homework question.  

\[
\Parens{\begin{array}{rrrr|c}
2 & -1 & 3 & 7 & 15\\
4 & 4 & 0 & 7 & 11\\
2 & 1 & 1 & 3 & 7\\
6 & 5 & 4 & 17 & 31 
\end{array}}
\]

The scales are as follows: $s_1 = 7, s_2 = 7, s_3 = 3, s_4 = 17.$

We pick $\ell_1.$ It should be the index which maximizes $\abs{a_{i1}}/s_i.$
These values are:
\[
\frac{2}{7},\frac{4}{7},\frac{2}{3},\frac{6}{17}.
\]
We pick $\ell_1 = 3,$ and pivot:

\[
\Parens{\begin{array}{rrrr|c}
0 & -2 & 2 & 4 & 8\\
0 & 2 & -2 & 1 & -3\\
2 & 1 & 1 & 3 & 7\\
0 & 2 & 1 & 8 & 10 
\end{array}}
\]

We pick $\ell_2.$ It should \emph{not} be $3,$ and should be the index which maximizes $\abs{a_{i2}}/s_i.$
These values are:
\[
\frac{2}{7},\frac{2}{7},\frac{2}{17}.
\]
We have a tie.  In this case we pick the second row, \ie $\ell_2 = 2.$ We
pivot:

\[
\Parens{\begin{array}{rrrr|c}
0 & 0 & 0 & 5 & 5\\
0 & 2 & -2 & 1 & -3\\
2 & 1 & 1 & 3 & 7\\
0 & 0 & 3 & 7 & 13 
\end{array}}
\]

The matrix is in permuted upper triangular form.  We could proceed, but would
get a zero multiplier, and no changes would occur.  

If we did proceed we would have $\ell_3 = 4.$  Then $\ell_4 = 1.$ Our row
permutation is $3,2,4,1.$  When we do back substitution, we work in this order
\emph{reversed} on the rows, solving $x_4,$ then $x_3,x_2,x_1.$

We get $x_4 = 1,$ so 
\begin{align*}
x_3 &= \oneby{3}\Parens{13 - 7*1} = 2\\
x_2 &= \oneby{2}\Parens{-3 - 1*1 + 2*2} = 0\\
x_1 &= \oneby{2}\Parens{7 - 3*1 - 1*2 - 1*0} = 1
\end{align*}

%UNFOLD
%%%%%%%%%%%%%%%%%%%%%%%%%%%%%%%%%%%%%%%%%%%%%%%
\subsection{Another Example and A Real Algorithm}%FOLDUP

Sometimes we want to solve 
\[\Mtx{A}\vect{x} = \vect{b}\]
for a number of different vectors \vect{b}.  It turns out we can run G.E. on
the matrix \Mtx{A} alone and come up with all the multipliers, which can then
be used multiple times on different vectors \vect{b}.  We illustrate with an
example:
\[
\Mtx{M_0} = \Parens{\begin{array}{rrrr}
1 & 2 & 4 & 1 \\
4 & 2 & 1 & 2 \\
2 & 1 & 2 & 3 \\
1 & 3 & 2 & 1 
\end{array}},
\quad 
\vect{\ell} = \Bracks{\begin{array}{c}1 \\ 2 \\ 3 \\ 4\end{array}}.
\]

The scale vector is $\vect{s} = \trans{\Bracks{\begin{array}{cccc}4& 4& 3&
3\end{array}}}$.  

%FIX: we should have the multipliers, not their negatives
%in the boxes. this makes the following section easier.
%
Our scale choices are $\frac{1}{4},\frac{4}{4},\frac{2}{3},\frac{1}{3}.$  We
choose $\ell_1 = 2,$ and swap $\ell_1,\ell_2.$  In the places where there would
be zeros in the real matrix, we will put the multipliers.  We will illustrate
them here boxed:
\[
\Mtx{M_1} = \Parens{\begin{array}{rrrr}
\boxed{\oneby{4}} & \frac{3}{2} & \frac{15}{4} & \half \\
4 & 2 & 1 & 2 \\
\boxed{\oneby{2}} & 0 & \frac{3}{2} & 2 \\
\boxed{\oneby{4}} & \frac{5}{2} & \frac{7}{4} & \half 
\end{array}},
\quad 
\vect{\ell} = \Bracks{\begin{array}{c}2 \\ 1 \\ 3 \\ 4\end{array}}.
\]

Our scale choices are $\frac{3}{8},\frac{0}{3},\frac{5}{6}.$  We
choose $\ell_2 = 4,$ and so swap $\ell_2,\,\ell_4$:
\[
\Mtx{M_2} = \Parens{\begin{array}{rrrr}
\boxed{\oneby{4}} & \boxed{\frac{3}{5}} & \frac{27}{10} & \oneby{5}\\
4 & 2 & 1 & 2 \\
\boxed{\oneby{2}} & \boxed{0} & \frac{3}{2} & 2 \\
\boxed{\oneby{4}} & \frac{5}{2} & \frac{7}{4} & \half 
\end{array}},
\quad 
\vect{\ell} = \Bracks{\begin{array}{c}2 \\ 4 \\ 3 \\ 1\end{array}}.
\]

Our scale choices are $\frac{27}{40},\frac{1}{2}.$  We
choose $\ell_3 = 1,$ and so swap $\ell_3,\ell_4$:
\[
\Mtx{M_3} = \Parens{\begin{array}{rrrr}
\boxed{\oneby{4}} & \boxed{\frac{3}{5}} & \frac{27}{10} & \oneby{5}\\
4 & 2 & 1 & 2 \\
\boxed{\oneby{2}} & \boxed{0} & \boxed{\frac{5}{9}} & \frac{17}{9} \\
\boxed{\oneby{4}} & \frac{5}{2} & \frac{7}{4} & \half 
\end{array}},
\quad 
\vect{\ell} = \Bracks{\begin{array}{c}2 \\ 4 \\ 1 \\ 3\end{array}}.
\]

Now suppose we had to solve the linear system for 
$\vect{b} = \trans{\Bracks{\begin{array}{cccc}-1& 8& 2& 1\end{array}}}$.  

We scale $\vect{b}$ by the multipliers in order: $\ell_1 = 2,$ so, we sweep
through the first column of \Mtx{M_3}, picking off the boxed numbers (your
computer doesn't really have boxed variables), and scaling \vect{b}
appropriately:
\[\Bracks{\begin{array}{c}-1 \\ 8 \\ 2 \\ 1\end{array}} 
\Rightarrow \Bracks{\begin{array}{c} -3 \\ 8 \\ -2 \\ -1\end{array}} 
\]
This continues:
\[\Bracks{\begin{array}{c} -3 \\ 8 \\ -2 \\ -1\end{array}} 
\Rightarrow \Bracks{\begin{array}{c} -\frac{12}{5} \\ 8 \\ -2 \\ -1\end{array}} 
\Rightarrow \Bracks{\begin{array}{c} -\frac{12}{5} \\ 8 \\ -\frac{2}{3} \\-1\end{array}} 
\]

We then perform a permuted backwards substitution on the augmented system
\[
\Parens{\begin{array}{*{4}{c}|c}
0 & 0 & \frac{27}{10} & \oneby{5} & -\frac{12}{5}\\
4 & 2 & 1 & 2 & 8\\
0 & 0 & 0 & \frac{17}{9} & -\frac{2}{3}\\
0 & \frac{5}{2} & \frac{7}{4} & \half & -1
\end{array}}
\]
This proceeds as 
\begin{align*}
x_4 &= \frac{-2}{3}\frac{9}{17} = \frac{-6}{17}\\
x_3 &= \frac{10}{27}\Parens{-\frac{12}{5} - \oneby{5}\frac{-6}{17}} = \ldots\\
x_2 &= \frac{2}{5}\Parens{-1 - \half \frac{-6}{17} - \frac{7}{4} x_3} = \ldots\\
x_1 &= \frac{1}{4}\Parens{8 - 2 \frac{-6}{17} - x_3 - 2 x_2} = \ldots
\end{align*}
Fill in your own values here.

%UNFOLD
%UNFOLD
%%%%%%%%%%%%%%%%%%%%%%%%%%%%%%%%%%%%%%%%%%%%%%%
\section{LU Factorization}%FOLDUP
\label{sec:LUfact}
\depson{sec:gaussianP}{sec:LUfact}

We examined G.E. to solve the system
\[\Mtx{A} \vect{x} = \vect{b},\]
where \Mtx{A} is a matrix:
\[
\Mtx{A} = \Bracks{\begin{array}{ccccc}
a_{11} & a_{12} & a_{13} & \cdots & a_{1n}\\
a_{21} & a_{22} & a_{23} & \cdots & a_{2n}\\
a_{31} & a_{32} & a_{33} & \cdots & a_{3n}\\
\vdots & \vdots & \vdots & \ddots & \vdots\\
a_{n1} & a_{n2} & a_{n3} & \cdots & a_{nn}
\end{array}}.
\]
We want to show that G.E. actually factors \Mtx{A} into lower and upper
triangular parts, that is $\Mtx{A} = \Mtx{L}\Mtx{U},$ where
\[
\Mtx{L} = \Bracks{\begin{array}{ccccc}
1 & 0 & 0 & \cdots & 0\\
\ell_{21} & 1 & 0 & \cdots & 0\\
\ell_{31} & \ell_{32} & 1 & \cdots & 0\\
\vdots & \vdots & \vdots & \ddots & \vdots\\
\ell_{n1} & \ell_{n2} & \ell_{n3} & \cdots & 1
\end{array}}, \qquad 
\Mtx{U} = \Bracks{\begin{array}{ccccc}
u_{11} & u_{12} & u_{13} & \cdots & u_{1n}\\
0 & u_{22} & u_{23} & \cdots & u_{2n}\\
0 & 0 & u_{33} & \cdots & u_{3n}\\
\vdots & \vdots & \vdots & \ddots & \vdots\\
0 & 0 & 0 & \cdots & u_{nn}
\end{array}}.
\]
We call this a \emph{LU Factorization} of \Mtx{A}.
\index{LU Factorization}


%%%%%%%%%%%%%%%%%%%%%%%%%%%%%%%%%%%%%%%%%%%%%%%
\subsection{An Example}%FOLDUP

We consider solution of the following augmented form:
\begin{eqnarray}
\Parens{\begin{array}{rrrr|c}
2 & 1 & 1 & 3 & 7\\
4 & 4 & 0 & 7 & 11\\
6 & 5 & 4 & 17 & 31\\
2 & -1 & 0 & 7 & 15
\end{array}}\label{eqn:ourEx}
\end{eqnarray}
The \naive G.E. reduces this to 
\[
\Parens{\begin{array}{rrrr|c}
2 & 1 & 1 & 3 & 7\\
0 & 2 & -2 & 1 & -3\\
0 & 0 & 3 & 7 & 13\\
0 & 0 & 0 & 12 & 18
\end{array}}
\]
We are going to run the \naive G.E., and see how it is a LU Factorization.
Since this is the \naive version, we first pivot on the first row.  Our
multipliers are $2,3,1.$  We pivot to get

\[
\Parens{\begin{array}{rrrr|c}
2 & 1 & 1 & 3 & 7\\
0 & 2 & -2 & 1 & -3\\
0 & 2 & 1 & 8 & 10\\
0 & -2 & -1 & 4 & 8
\end{array}}
\]
Careful inspection shows that we've merely multiplied \Mtx{A} and \vect{b} by a
lower triangular matrix \Mtx{M_1}:
\[
\Mtx{M_1} = \Bracks{\begin{array}{rrrr}
1 & 0 & 0 & 0\\
-2 & 1 & 0 & 0\\
-3 & 0 & 1 & 0\\
-1 & 0 & 0 & 1
\end{array}}
\]
The entries in the first column are the negative e.r.o. multipliers for each row.  Thus after
the first pivot, it is like we are solving the system
\[\Mtx{M_1} \Mtx{A} \vect{x} = \Mtx{M_1} \vect{b}.\]
We pivot on the second row to get:
\[
\Parens{\begin{array}{rrrr|c}
2 & 1 & 1 & 3 & 7\\
0 & 2 & -2 & 1 & -3\\
0 & 0 & 3 & 7 & 13\\
0 & 0 & -3 & 5 & 5
\end{array}}
\]
The multipliers are $1,-1.$  We can view this pivot as a multiplication by
$\Mtx{M_2},$ with 
\[
\Mtx{M_2} = \Bracks{\begin{array}{rrrr}
1 & 0 & 0 & 0\\
0 & 1 & 0 & 0\\
0 & -1 & 1 & 0\\
0 & 1 & 0 & 1
\end{array}}
\]
We are now solving
\[\Mtx{M_2} \Mtx{M_1} \Mtx{A} \vect{x} = \Mtx{M_2} \Mtx{M_1} \vect{b}.\]

We pivot on the third row, with a multiplier of $-1.$  Thus we get
\[
\Parens{\begin{array}{rrrr|c}
2 & 1 & 1 & 3 & 7\\
0 & 2 & -2 & 1 & -3\\
0 & 0 & 3 & 7 & 13\\
0 & 0 & 0 & 12 & 18
\end{array}}
\]
We have multiplied by $\Mtx{M_3}:$ 
\[
\Mtx{M_3} = \Bracks{\begin{array}{rrrr}
1 & 0 & 0 & 0\\
0 & 1 & 0 & 0\\
0 & 0 & 1 & 0\\
0 & 0 & 1 & 1
\end{array}}
\]
We are now solving
\[\Mtx{M_3}\Mtx{M_2}\Mtx{M_1}\Mtx{A} \vect{x} = \Mtx{M_3}\Mtx{M_2}\Mtx{M_1} \vect{b}.\]
But we have an upper triangular form, that is, if we let
\[
\Mtx{U} = \Bracks{\begin{array}{rrrr}
2 & 1 & 1 & 3 \\
0 & 2 & -2 & 1 \\
0 & 0 & 3 & 7 \\
0 & 0 & 0 & 12 
\end{array}}
\]
Then we have
\begin{align*}
\Mtx{M_3}\Mtx{M_2}\Mtx{M_1}\Mtx{A} &= \Mtx{U},\\
\Mtx{A} &= \invs{\Parens{\Mtx{M_3}\Mtx{M_2}\Mtx{M_1}}} \Mtx{U},\\
\Mtx{A} &= \invs{\Mtx{M_1}}\invs{\Mtx{M_2}}\invs{\Mtx{M_3}} \Mtx{U},\\
\Mtx{A} &= \Mtx{L} \Mtx{U}.
\end{align*}
We are hoping that \Mtx{L} is indeed lower triangular, and has ones on the
diagonal.  It turns out that the inverse of each \Mtx{M_i} matrix has a nice
form (See \chexref{invM}).  We write them here:
\begin{align*}
\Mtx{L} &=
\Bracks{\begin{array}{rrrr}
1 & 0 & 0 & 0\\
2 & 1 & 0 & 0\\
3 & 0 & 1 & 0\\
1 & 0 & 0 & 1
\end{array}}
\Bracks{\begin{array}{rrrr}
1 & 0 & 0 & 0\\
0 & 1 & 0 & 0\\
0 & 1 & 1 & 0\\
0 & -1 & 0 & 1
\end{array}}
\Bracks{\begin{array}{rrrr}
1 & 0 & 0 & 0\\
0 & 1 & 0 & 0\\
0 & 0 & 1 & 0\\
0 & 0 & -1 & 1
\end{array}}\\
&=
\Bracks{\begin{array}{rrrr}
1 & 0 & 0 & 0\\
2 & 1 & 0 & 0\\
3 & 1 & 1 & 0\\
1 & -1 & -1 & 1
\end{array}}
\end{align*}
This is really crazy: the matrix \Mtx{L} looks to be composed of ones on the
diagonal and multipliers under the diagonal.

Now we check to see if we made any mistakes:
\begin{align*}
\Mtx{L} \Mtx{U} &=
\Bracks{\begin{array}{rrrr}
1 & 0 & 0 & 0\\
2 & 1 & 0 & 0\\
3 & 1 & 1 & 0\\
1 & -1 & -1 & 1
\end{array}}
\Bracks{\begin{array}{rrrr}
2 & 1 & 1 & 3 \\
0 & 2 & -2 & 1 \\
0 & 0 & 3 & 7 \\
0 & 0 & 0 & 12 
\end{array}}\\
&=
\Bracks{\begin{array}{rrrr}
2 & 1 & 1 & 3 \\
4 & 4 & 0 & 7 \\
6 & 5 & 4 & 17 \\
2 & -1 & 0 & 7 
\end{array}} = \Mtx{A}.
\end{align*}

%UNFOLD
%%%%%%%%%%%%%%%%%%%%%%%%%%%%%%%%%%%%%%%%%%%%%%%
\subsection{Using LU Factorizations}%FOLDUP

We see that the G.E. algorithm can be used to actually calculate the LU
factorization.  We will look at this in more detail in another example.
We now examine how we can use the LU factorization to solve the equation
\[\Mtx{A} \vect{x} = \vect{b},\]
Since we have $\Mtx{A} = \Mtx{L}\Mtx{U},$ we first solve
\[\Mtx{L} \vect{z} = \vect{b},\]
then solve
\[\Mtx{U} \vect{x} = \vect{z}.\]
Since \Mtx{L} is lower triangular, we can solve for \vect{z} with a
\emph{forward} substitution.  Similarly, since \Mtx{U} is upper triangular, we
can solve for \vect{x} with a back substitution.  We drag out the previous
example (which we never got around to solving):
\[
\Parens{\begin{array}{rrrr|c}
2 & 1 & 1 & 3 & 7\\
4 & 4 & 0 & 7 & 11\\
6 & 5 & 4 & 17 & 31\\
2 & -1 & 0 & 7 & 15
\end{array}}
\]
We had found the LU factorization of \Mtx{A} as 

\begin{align*}
\Mtx{A} &=
\Bracks{\begin{array}{rrrr}
1 & 0 & 0 & 0\\
2 & 1 & 0 & 0\\
3 & 1 & 1 & 0\\
1 & -1 & -1 & 1
\end{array}}
\Bracks{\begin{array}{rrrr}
2 & 1 & 1 & 3 \\
0 & 2 & -2 & 1 \\
0 & 0 & 3 & 7 \\
0 & 0 & 0 & 12 
\end{array}}
\end{align*}
So we solve 
\[
\Bracks{\begin{array}{rrrr}
1 & 0 & 0 & 0\\
2 & 1 & 0 & 0\\
3 & 1 & 1 & 0\\
1 & -1 & -1 & 1
\end{array}}\vect{z} = 
\Bracks{\begin{array}{r} 7\\ 11\\ 31\\ 15 \end{array}}
\]
We get
\[
\vect{z} = \Bracks{\begin{array}{r} 7\\ -3\\ 13\\ 18 \end{array}}
\]
Now we solve
\[
\Bracks{\begin{array}{rrrr}
2 & 1 & 1 & 3 \\
0 & 2 & -2 & 1 \\
0 & 0 & 3 & 7 \\
0 & 0 & 0 & 12 
\end{array}}\vect{x} = \Bracks{\begin{array}{r} 7\\ -3\\ 13\\ 18 \end{array}}
\]
We get the ugly solution
\[
\vect{z} = \Bracks{\begin{array}{c} \frac{37}{24}\\ \frac{-17}{12}\\
\frac{5}{6}\\ \frac{3}{2} \end{array}}
\]

%UNFOLD
%%%%%%%%%%%%%%%%%%%%%%%%%%%%%%%%%%%%%%%%%%%%%%%
\subsection{Some Theory}%FOLDUP

We aren't doing much proving here.  The following theorem has an ugly proof in
the Cheney \& Kincaid \cite{KdCw1999}.
\begin{bktheorem}
If \Mtx{A} is an $n \times n$ matrix, and \naive Gaussian Elimination does not
encounter a zero pivot, then the algorithm generates a LU factorization of
\Mtx{A}, where \Mtx{L} is the lower triangular part of the output matrix, and
\Mtx{U} is the upper triangular part.
\end{bktheorem}

This theorem relies on us using the fancy version of G.E., which saves the
multipliers in the spots where there should be zeros.  If correctly
implemented, then, \Mtx{L} is the lower triangular part but with ones put on
the diagonal.

This theorem is proved in Cheney \& Kincaid \cite{KdCw1999}.  This appears to
me to be a case of something which can be better illustrated with an example or
two and some informal investigation.  The proof is an unillustrating
index-chase--read it at your own risk.

%UNFOLD
%%%%%%%%%%%%%%%%%%%%%%%%%%%%%%%%%%%%%%%%%%%%%%%
\subsection{Computing Inverses}%FOLDUP

We consider finding the inverse of \Mtx{A}.  Since
\[\Mtx{A} \invs{\Mtx{A}} = \Mtx{I},\]
then the \kth{j} column of the inverse \invs{\Mtx{A}} solves the equation
\[\Mtx{A} \vect{x} = \vect{e_j},\]
where \vect{e_j} is the column matrix of all zeros, but with a one in the \kth{j}
position.

Thus we can find the inverse of \Mtx{A} by running $n$ linear solves.
Obviously we are only going to run G.E. once, to put the matrix in LU form,
then run $n$ solves using forward and backward substitutions.

%UNFOLD
%UNFOLD
%%%%%%%%%%%%%%%%%%%%%%%%%%%%%%%%%%%%%%%%%%%%%%%
\section{Iterative Solutions}%FOLDUP
\label{sec:iterative}

\depson{sec:eiegenvalues}{sec:iterative}

Recall we are trying to solve 
\[\Mtx{A} \vect{x} = \vect{b}.\label{eqn:theeq}\]
We examine the computational cost of Gaussian Elimination 
to motivate the search for an alternative algorithm.

%%%%%%%%%%%%%%%%%%%%%%%%%%%%%%%%%%%%%%%%%%%%%%%
\subsection{An Operation Count for Gaussian Elimination}%FOLDUP

We consider the number of floating point operations (``flops'') required to
solve the system $\Mtx{A} \vect{x} = \vect{b}.$  Gaussian Elimnation first
uses row operations to transform the problem into an equivalent problem of the
form $\Mtx{U} \vect{x} = \vect{b'},$ where \Mtx{U} is upper triangular.   Then
back substitution is used to solve for \vect{x}.

First we look
at how many floating point operations are required to reduce 
\[
\Parens{\begin{array}{ccccc|c}
a_{11} & a_{12} & a_{13} & \cdots & a_{1n} & b_1\\
a_{21} & a_{22} & a_{23} & \cdots & a_{2n} & b_2\\
a_{31} & a_{32} & a_{33} & \cdots & a_{3n} & b_3\\
\vdots & \vdots & \vdots & \ddots & \vdots &\\
a_{n1} & a_{n2} & a_{n3} & \cdots & a_{nn} & b_n
\end{array}}
\]
to 
\[
\Parens{\begin{array}{ccccc|c}
a_{11} & a_{12} & a_{13} & \cdots & a_{1n} & b_1\\
0 & a'_{22} & a'_{23} & \cdots & a'_{2n} & b'_2\\
0 & a'_{32} & a'_{33} & \cdots & a'_{3n} & b'_3\\
\vdots & \vdots & \vdots & \ddots & \vdots &\\
0 & a'_{n2} & a'_{n3} & \cdots & a'_{nn} & b'_n
\end{array}}
\]

First a multiplier is computed for each row.  Then in each row the algorithm
performs $n$ multiplies and $n$ adds.  This gives a total of $(n-1) + (n-1)n$
multiplies (counting in the computing of the multiplier in each of the $(n-1)$
rows) and $(n-1)n$ adds.  In total this is $2n^2 - n - 1$ floating point
operations to do a single pivot on the $n$ by $n$ system. 

Then this has to be done recursively on the lower right subsystem,
which is an $(n-1)$ by $(n-1)$ system.  This requires $2(n-1)^2 - (n-1) - 1$
operations.  Then this has to be done on the next subsystem, requiring
$2(n-2)^2 - (n-2) - 1$ operations, and so on.

In total, then, we use $I_n$ total floating point operations, with
\[I_n = 2 \sum_{j=1}^n j^2  - \sum_{j=1}^n j - \sum_{j=1}^n 1.\]

Recalling that 
\[\sum_{j=1}^n j^2 = \oneby{6}(n)(n+1)(2n+1),\quad\text{and}\quad
\sum_{j=1}^n j = \oneby{2}(n)(n+1),\]
We find that
\[I_n = \oneby{6}(4n-1)n(n+1) - n \approx \frac{2}{3}n^3.\]

Now consider the costs of back substitution.  To solve 
\[
\Parens{\begin{array}{ccccc|c}
a_{11} & \cdots & a_{1,n-2} & a_{1,n-1} & a_{1n} & b_1\\
\vdots & \ddots & \vdots & \vdots & \vdots & \vdots\\
 0     & \cdots & a_{n-2,n-2} & a_{n-2,n-1} & a_{n-2,n} & b_{n-2}\\
 0     & \cdots &    0   & a_{n-1,n-1} & a_{n-1,n} & b_{n-1}\\
 0     & \cdots &    0   & 0      & a_{nn} & b_n
\end{array}}
\]
for $x_n$ requires only a single division.  Then to solve for $x_{n-1}$ we
compute
\[x_{n-1} = \oneby{a_{n-1,n-1}} \Bracks{b_{n-1} - a_{n-1,n} x_{n}},\]
and requires 3 flops.  Similarly, solving for $x_{n-2}$ requires $5$ flops.  
Thus in total back substitution requires $B_n$ total floating point operations
with \[B_n = \sum_{j=1}^n 2j - 1 = n(n-1) - n = n(n-2) \approx n^2\]
%UNFOLD
%%%%%%%%%%%%%%%%%%%%%%%%%%%%%%%%%%%%%%%%%%%%%%%
\subsection{Dividing by Multiplying}%FOLDUP
We saw that Gaussian Elimination requires around $\frac{2}{3}n^3$ operations
just to find the LU factorization, then about $n^2$ operations to solve the
system, when \Mtx{A} is $n\times n$.  When $n$ is large, this may take too long
to be practical.  Additionally, if \Mtx{A} is sparse (has few nonzero elements
per row), we would like the complexity of our computations to scale with the
sparsity of \Mtx{A}.  Thus we look for an alternative algorithm. 

First we consider the simplest case, $n=1.$  Suppose we are to solve the
equation
\[A x = b.\]
for scalars $A,b.$  We solve this by
\[x = \oneby{A} b = \oneby{\omega A} \omega b =
\oneby{1 - (1 - \omega A)} \omega b 
= \oneby{1-r} \omega b,\]
where $\omega \ne 0$ is some real number chosen to ``weight'' the problem
appropriately, and $r = 1 - \omega A.$  
Now suppose that $\omega$ is chosen such that $\abs{r} < 1.$ This can be done
so long as $A \ne 0,$ which would have been a problem anyway.
Now use the geometric expansion:
\[\oneby{1 - r} = 1 + r + r^2 + r^3 + \ldots\]
Because of the assumption $\abs{r} < 1,$ the terms $r^n$ converge
to zero as $n\to\infty.$  This gives the approximate solution to our one
dimensional problem as 
\begin{align*}
x &\approx \Bracks{ 1 + r + r^2 + r^3 + \ldots + r^k }\omega b\\
  &= \omega b + \Bracks{ r + r^2 + r^3 + \ldots + r^k }\omega b\\
  &= \omega b + r \Bracks{ 1 + r + r^2 + \ldots + r^{k-1} }\omega b
\end{align*}
This suggests an \emph{iterative} approach to solving $Ax=b.$  First let
$x^{(0)} = \omega b,$ then let 
\[x^{(k)} = \omega b + r x^{(k-1)}.\]
The iterates $x^{(k)}$ will converge to the solution of $Ax=b$ if $\abs{r} <
1.$

You should now convince yourself that because $r^n \to 0,$ that the
choice of the initial iterate $x^{(0)}$ was immaterial, \ie that under
\emph{any} choice of initial iterate convergence is guaranteed.

We now translate this scalar result into the vector case. The algorithm
proceeds as follows:  first fix some initial estimate of the solution,
\thex{0}.  A good choice might be $\omega \vect{b},$ but this is not necessary.
Then calculate successive approximations to the actual solution by updates of
the form 
\[\thex{k} = \omega\vect{b} + \Parens{\Mtx{I} - \omega\Mtx{A}}\thex{k-1}.\]

It turns out that we can consider a slightly more general form of the
algorithm, one in which successive iterates are defined \emph{implicitly}.
That is we consider iterates of the form
\begin{empheq}[innerbox=\widefbox]{equation}
\Mtx{Q}\thex{k+1} = \Parens{\Mtx{Q} - \omega \Mtx{A}}\thex{k} + \omega \vect{b},
\label{eqn:iterativeupdate}
\end{empheq}
for some matrix \Mtx{Q}, and some scaling factor $\omega.$  Note that this
update relies on vector additions and possibly by premultiplication of a vector
by \Mtx{A} or \Mtx{Q}.  In the case where these two matrices are sparse, such
an update can be relatively cheap.

Now suppose that as $k \to \infty,$ \thex{k} converges to some vector \thexc,
which is a fixed point of the iteration.
Then 
\begin{align*}
\Mtx{Q}\thexc &= \Parens{\Mtx{Q} - \omega\Mtx{A}}\thexc + \omega \vect{b},\\
\Mtx{Q}\thexc &= \Mtx{Q}\thexc - \omega\Mtx{A}\thexc + \omega\vect{b},\\
\omega\Mtx{A}\thexc &= \omega\vect{b},\\
\Mtx{A}\thexc &= \vect{b}.
\end{align*}

We have some freedom in choosing \Mtx{Q}, but there are two considerations we
should keep in mind:
\begin{compactenum}
\item Choice of \Mtx{Q} affects convergence and speed of convergence of the
method.  In particular, we want \Mtx{Q} to be similar to \Mtx{A}.
\item Choice of \Mtx{Q} affects ease of computing the update.  That is, given 
\[\vect{z} = \Parens{\Mtx{Q} - \Mtx{A}}\thex{k} + \vect{b},\]
we should pick \Mtx{Q} such that the equation
\[\Mtx{Q} \thex{k+1} = \vect{z}\] is easy to solve exactly.
\end{compactenum}
These two goals conflict with each other.  At one end of the spectrum is the
so-called ``impossible iteration,'' at the other is the Richardsons.
%UNFOLD
%%%%%%%%%%%%%%%%%%%%%%%%%%%%%%%%%%%%%%%%%%%%%%%
\subsection{Impossible Iteration}%FOLDUP
I made up the term ``impossible iteration.''  But consider the method which
takes \Mtx{Q} to be \Mtx{A}.  This seems to be the best choice for satisfying
the first goal.  Letting $\omega = 1,$ our method becomes
\[
\Mtx{A}\thex{k+1} = \Parens{\Mtx{A} - \Mtx{A}}\thex{k} + \vect{b} = \vect{b}.
\]
This method should clearly converge in one step.  However, the second goal is
totally ignored.  Indeed, we are considering iterative methods because we
cannot easily solve this linear equation in the first place.
%UNFOLD
%%%%%%%%%%%%%%%%%%%%%%%%%%%%%%%%%%%%%%%%%%%%%%%
\subsection{Richardson Iteration}%FOLDUP

At the other end of the spectrum is the Richardson Iteration, which chooses
\Mtx{Q} to be the identity matrix.  Solving the system \[\Mtx{Q} \thex{k+1} =
\vect{z}\] is trivial: we just have $\thex{k+1} = \vect{z}.$

\begin{bkexprob}\label{bkexp:richbadomega}%FOLDUP
Use Richardson Iteration with $\omega = 1$ on the system 
\[
\Mtx{A} = \Bracks{\begin{array}{rrr}
6 & 1 & 1 \\
2 & 4 & 0 \\
1 & 2 & 6 
\end{array}},\qquad 
\Mtx{b} = \Bracks{\begin{array}{c} 12\\0\\6 \end{array}}.
\]
\begin{bksolution}
We let
\[
\Mtx{Q} = \Bracks{\begin{array}{rrr}
1 & 0 & 0 \\
0 & 1 & 0 \\
0 & 0 & 1 
\end{array}},\qquad 
\Parens{\Mtx{Q} - \Mtx{A}} = \Bracks{\begin{array}{rrr}
-5 & -1 & -1 \\
-2 & -3 & 0 \\
-1 & -2 & -5 
\end{array}}.
\]
We start with an arbitrary \thex{0}, say $\thex{0} = \trans{\Bracks{2\; 2\; 2}}.$
We get $\thex{1} = \trans{\Bracks{-2\;-10\;-10}},$ and $\thex{2} =
\trans{\Bracks{42\;34\;78}}.$  

Note the real solution is $\vect{x} = \trans{\Bracks{2\;-1\;1}}.$ 
The Richardson Iteration does not appear to converge for this example, unfortunately.
\end{bksolution}
\end{bkexprob}
%UNFOLD
\begin{bkexprob}\label{bkexp:richgoodomega}%FOLDUP
Apply Richardson Iteration with $\omega = 1/6$ on the previous system.
\begin{bksolution}
Our iteration becomes
\[
\thex{k+1} = 
\Bracks{\begin{array}{rrr}
0 & -1/6 & -1/6 \\
-1/3 & 1/3 & 0 \\
-1/6 & -1/3 & 0 
\end{array}}
\thex{k} + 
\Bracks{\begin{array}{c} 2\\0\\1 \end{array}}.
\]

We start with the same \thex{0} as previously, $\thex{0} = \trans{\Bracks{2\; 2\; 2}}.$
We get $\thex{1} = \trans{\Bracks{4/3 \; 0 \;0}},$ 
$\thex{2} = \trans{\Bracks{2 \; -4/9 \; 7/9}},$ and finally
$\thex{12} = \trans{\Bracks{2 \; -0.99998 \; 0.99998}}.$

Thus, the choice of $\omega$ has some affect on convergence.
\end{bksolution}
\end{bkexprob}%UNFOLD

We can rethink the Richardson Iteration as 
\[
\thex{k+1} = \Parens{\Mtx{I} - \omega\Mtx{A}}\thex{k} + \omega \vect{b} = \thex{k} +
\omega \Parens{\vect{b} - \Mtx{A}\thex{k}}.
\]
Thus at each step we are adding some scaled version of the \emph{residual},
defined as $\vect{b} - \Mtx{A}\thex{k}$, to the iterate.

%UNFOLD
%%%%%%%%%%%%%%%%%%%%%%%%%%%%%%%%%%%%%%%%%%%%%%%
\subsection{Jacobi Iteration}%FOLDUP

The Jacobi Iteration chooses \Mtx{Q} to be the matrix consisting
of the diagonal of \Mtx{A}.  This is more similar to \Mtx{A} than the identity
matrix, but nearly as simple to invert.

\begin{bkexprob}%FOLDUP
Use Jacobi Iteration, with $\omega = 1,$ to solve the system 
\[
\Mtx{A} = \Bracks{\begin{array}{rrr}
6 & 1 & 1 \\
2 & 4 & 0 \\
1 & 2 & 6 
\end{array}},\qquad 
\Mtx{b} = \Bracks{\begin{array}{c} 12\\0\\6 \end{array}}.
\]
\begin{bksolution}
We let
\[
\Mtx{Q} = \Bracks{\begin{array}{rrr}
6 & 0 & 0 \\
0 & 4 & 0 \\
0 & 0 & 6 
\end{array}},\qquad 
\Parens{\Mtx{Q} - \Mtx{A}} = \Bracks{\begin{array}{rrr}
0 & -1 & -1 \\
-2 & 0 & 0 \\
-1 & -2 & 0 
\end{array}},\qquad 
\invs{\Mtx{Q}} = \Bracks{\begin{array}{rrr}
\oneby{6} & 0 & 0 \\
0 & \oneby{4} & 0 \\
0 & 0 & \oneby{6}
\end{array}}.
\]
We start with an arbitrary \thex{0}, say $\thex{0} = \trans{\Bracks{2\; 2\; 2}}.$
We get $\thex{1} = \trans{\Bracks{\frac{4}{3}\;-1\;0}}.$ 
Then $\thex{2} = \trans{\Bracks{\frac{13}{6}\;-\frac{2}{3}\;\frac{10}{9}}}.$ 
Continuing, we find that $\thex{5} \approx
\trans{\Bracks{1.987\;-1.019\;0.981}}.$

Note the real solution is $\vect{x} = \trans{\Bracks{2\;-1\;1}}.$ 
\end{bksolution}
\end{bkexprob}
%UNFOLD
%
There is an alternative way to describe the Jacobi Iteration for $\omega = 1.$
By considering the update elementwise, we see that the operation can be
described by
\[\thexi{k+1}{j} = \oneby{a_{jj}} \Parens{b_j - \sum_{i=1, i\ne j}^{n} a_{ji} \thexi{k}{i}}.\]
Thus an update takes less than $2n^2$ operations.  In fact, if \Mtx{A} is
sparse, with less than $k$ nonzero entries per row, the update should take less
than $2nk$ operations.

%UNFOLD
%%%%%%%%%%%%%%%%%%%%%%%%%%%%%%%%%%%%%%%%%%%%%%%
\subsection{Gauss Seidel Iteration}%FOLDUP

The Gauss Seidel Iteration chooses \Mtx{Q} to be lower triangular part of
\Mtx{A}, including the diagonal.  In this case solving the system \[\Mtx{Q} \thex{k+1} = \vect{z}\]
is performed by forward substitution.  Here the \Mtx{Q} is more like \Mtx{A}
than for Jacobi Iteration, but involves more work for inverting.

\begin{bkexprob}%FOLDUP
Use Gauss Seidel Iteration to again solve for
\[
\Mtx{A} = \Bracks{\begin{array}{rrr}
6 & 1 & 1 \\
2 & 4 & 0 \\
1 & 2 & 6 
\end{array}},\qquad 
\Mtx{b} = \Bracks{\begin{array}{c} 12\\0\\6 \end{array}}.
\]
\begin{bksolution}
We let
\[
\Mtx{Q} = \Bracks{\begin{array}{rrr}
6 & 0 & 0 \\
2 & 4 & 0 \\
1 & 2 & 6 
\end{array}},\qquad 
\Parens{\Mtx{Q} - \Mtx{A}} = \Bracks{\begin{array}{rrr}
0 & -1 & -1 \\
0 & 0 & 0 \\
0 & 0 & 0 
\end{array}}.
\]
We start with an arbitrary \thex{0}, say $\thex{0} = \trans{\Bracks{2\, 2\, 2}}.$
We get $\thex{1} = \trans{\Bracks{\frac{4}{3}\,-\frac{2}{3}\,1}}.$ 
Then $\thex{2} = \trans{\Bracks{\frac{35}{18}\,-\frac{35}{36}\,1}}.$ 

Already this is fairly close to the actual solution $\vect{x} = \trans{\Bracks{2\,-1\,\,1}}.$ 
\end{bksolution}
\end{bkexprob}
%UNFOLD
%

Just as with Jacobi Iteration, there is an easier way to describe the Gauss
Seidel Iteration.  In this case we will keep a single vector \vect{x} and overwrite it, element by element.  Thus for $j=\onetox{n},$ we set
\[
x_j \leftarrow \oneby{a_{jj}} \Parens{b_j - \sum_{i=1,i \ne j}^n a_{ji}x_i}.
\]
This looks exactly like the Jacobi update. However, in the sum on the right
there are some ``old'' values of $x_i$ and some ``new'' values; the new values
are those $x_i$ for which $i < j.$

Again this takes less than $2n^2$ operations. Or less than $2nk$ if \Mtx{A} is
sufficiently sparse.

An alteration of the Gauss Seidel Iteration is to make successive ``sweeps'' of
this redefinition, one for $j=\onetox{n},$ the next for $j=n,n-1,\ldots,2,1.$
This amounts to running Gauss Seidel once with \Mtx{Q} the lower triangular
part of \Mtx{A}, then running it with \Mtx{Q} the upper triangular part.  This
iterative method is known as ``red-black Gauss Seidel.''

%UNFOLD
%%%%%%%%%%%%%%%%%%%%%%%%%%%%%%%%%%%%%%%%%%%%%%%
%\subsection{Successive Overelaxation Iteration (SOR)}%FOLDUP
%
%%We can view Successive Overrelaxation (SOR) as a modification of the main method,%FOLDUP
%%using the Gauss Seidel Iteration matrix.  The SOR method is parametrized by $\omega \in \coinv{1}{2}.$ The iteration can be described as follows:
%%\begin{eqnarray*}
%%\Mtx{Q_g}\thez{k+1} &=& \Parens{\Mtx{Q_g} - \Mtx{A}}\thex{k} + \vect{b},\\
%%\thex{k+1} &=& \omega \thez{k+1} + \Parens{1-\omega} \thex{k}.
%%\end{eqnarray*}
%%Where \Mtx{Q_g} is the matrix of the Gauss Seidel Iteration.  In the case that
%%$\omega=1,$ we see that we regain the Gauss Seidel Iteration.  Otherwise we can
%%write
%%\begin{eqnarray*}
%%\thex{k+1} &=& \omega \invs{\Mtx{Q_g}} \Bracks{\Parens{\Mtx{Q_g} -
%%\Mtx{A}}\thex{k} + \vect{b}} + \Parens{1-\omega}\thex{k}.
%%\end{eqnarray*}
%%This works out to
%%\begin{eqnarray*}
%%\oneby{\omega}\Mtx{Q_g}\thez{k+1} &=& \Parens{\oneby{\omega}\Mtx{Q_g} - \Mtx{A}}\thex{k} + \vect{b},
%%\end{eqnarray*}
%%so the matrix \Mtx{Q} for the SOR iteration, call it \Mtx{Q_s} is defined as
%%\[\Mtx{Q_s} = \oneby{\omega}\Mtx{Q_g}.\]%UNFOLD
%
%The Successive Overelaxation (SOR) Iteration chooses \Mtx{Q} to be lower
%triangular part of \Mtx{A}, but with the diagonal divided by a relaxation
%factor, $\omega.$ In this case solving the system \[\Mtx{Q} \thex{k+1} = \vect{z}\]
%is performed by forward substitution.
%
%\begin{bkexample}
%Use SOR Iteration with $\omega = 1.5$ to again solve for
%\begin{eqnarray*}
%\Mtx{A} = \Bracks{\begin{array}{rrr}
%6 & 1 & 1 \\
%2 & 4 & 0 \\
%1 & 2 & 6 
%\end{array}},\qquad 
%\Mtx{b} = \Bracks{\begin{array}{c} 12\\0\\6 \end{array}}.
%\end{eqnarray*}
%We let
%\begin{eqnarray*}
%\Mtx{Q} = \Bracks{\begin{array}{rrr}
%4 & 0 & 0 \\
%2 & \frac{8}{3} & 0 \\
%1 & 2 & 4 
%\end{array}},\qquad 
%\Parens{\Mtx{Q} - \Mtx{A}} = \Bracks{\begin{array}{rrr}
%-2 & -1 & -1 \\
%0 & -\frac{4}{3} & 0 \\
%0 & 0 & -2
%\end{array}}.
%\end{eqnarray*}
%We start with an arbitrary \thex{0}, say $\thex{0} = \trans{\Bracks{2\, 2\, 2}}.$
%We get $\thex{1} = \trans{\Bracks{1\,-\frac{7}{4}\,\frac{9}{8}}}.$ 
%
%\end{bkexample}
%
%%UNFOLD
%%%%%%%%%%%%%%%%%%%%%%%%%%%%%%%%%%%%%%%%%%%%%%%
\subsection{Error Analysis}%FOLDUP

Suppose that \vect{x} is the solution to \eqnref{theeq}.  Define the error
vector:
\[\thee{k} = \thex{k} - \vect{x}.\]
Now notice that
\begin{align*}
\thex{k+1} &= \invs{\Mtx{Q}}\Parens{\Mtx{Q} - \omega\Mtx{A}}\thex{k} +
	\invs{\Mtx{Q}}\omega\vect{b},\\
\thex{k+1} &= \invs{\Mtx{Q}}\Mtx{Q} \thex{k} -
	\omega\invs{\Mtx{Q}}\Mtx{A}\thex{k} + \omega\invs{\Mtx{Q}}\Mtx{A}\vect{x},\\
\thex{k+1} &= \thex{k} - \omega\invs{\Mtx{Q}}\Mtx{A}\Parens{\thex{k} - \vect{x}},\\
\thex{k+1} - \vect{x} &= \thex{k} - \vect{x} - \omega\invs{\Mtx{Q}}\Mtx{A}\Parens{\thex{k} - \vect{x}},\\
\thee{k+1} &= \thee{k} - \omega\invs{\Mtx{Q}}\Mtx{A}\thee{k},\\
\thee{k+1} &= \Parens{\Mtx{I} - \omega\invs{\Mtx{Q}}\Mtx{A}}\thee{k}.
\end{align*}

Reusing this relation we find that
\begin{align*}
\thee{k} &= \Parens{\Mtx{I} - \omega\invs{\Mtx{Q}}\Mtx{A}}\thee{k-1},\\
				&= \Parens{\Mtx{I} - \omega\invs{\Mtx{Q}}\Mtx{A}}^2\thee{k-2},\\
				&= \Parens{\Mtx{I} - \omega\invs{\Mtx{Q}}\Mtx{A}}^k\thee{0}.
\end{align*}
We want to ensure that \thee{k+1} is ``smaller'' than \thee{k}.  
To do this we recall matrix and vector norms from \subsecref{matrixnorms}.
\begin{align*}
\dltwo{\thee{k}} &= \dltwo{\Parens{\Mtx{I} - \omega\invs{\Mtx{Q}}\Mtx{A}}^k\thee{0}}
\le \dltwo{\Mtx{I} - \omega\invs{\Mtx{Q}}\Mtx{A}}^k \dltwo{\thee{0}}.
\end{align*}
(See \bkexpref{mnormCS}.)

Thus our iteration converges (\thee{k} goes to the zero vector, \ie $\thex{k}
\to \thex{}$) if \[\dltwo{\Mtx{I} - \omega\invs{\Mtx{Q}}\Mtx{A}} < 1.\]
This gives the theorem:
\begin{bktheorem}\label{thm:iterconvg}
An iterative solution scheme converges for any starting \thex{0} if and only if
all eigenvalues of $\Mtx{I} - \omega\invs{\Mtx{Q}}\Mtx{A}$ are less than $1$ in
absolute value, \ie if and only if 
\[\dltwo{\Mtx{I} - \omega\invs{\Mtx{Q}}\Mtx{A}} < 1\]
\end{bktheorem}
Another way of saying this is ``the spectral radius of $\Mtx{I} -
\omega\invs{\Mtx{Q}}\Mtx{A}$ is less than $1$.''  

In fact, the \emph{speed} of convergence is decided by the spectral radius of
the matrix--convergence is faster for smaller values. 
Recall our introduction to iterative methods in the scalar case, where the
result relied on $\omega$ being chosen such that $\abs{1-\omega A} < 1$.  
You should now think about how eigenvalues generalize the absolute value of a
scalar, and how this relates to the norm of matrices.

Let \they{} be an eigenvector for $\invs{\Mtx{Q}}\Mtx{A},$ with corresponding
eigenvalue $\lambda.$  Then
\[
\Parens{\Mtx{I} - \omega\invs{\Mtx{Q}}\Mtx{A}}\they{} = \they{} - \omega
\invs{\Mtx{Q}}\Mtx{A}\they{} = \they{} - \omega\lambda \they{} =
\Parens{1-\omega\lambda}\they{}.
\]

This relation may allow us to pick the optimal $\omega$ for given
\Mtx{A},\Mtx{Q}.  It can also show us that sometimes no choice of $\omega$ will
give convergence of the method.  There are a number of different related
results that show when various methods will work for certain choices of
$\omega.$  We leave these to the exercises.

\begin{bkexprob}\label{bkexp:richbestomega}%FOLDUP
Find conditions on $\omega$ which guarantee convergence of Richardson's
Iteration for finding approximate iterative solutions to the system
$\Mtx{A}\vect{x} = \vect{b},$ where
\[
\Mtx{A} = \Bracks{\begin{array}{rrr}
6 & 1 & 1 \\
2 & 4 & 0 \\
1 & 2 & 6 
\end{array}},\qquad 
\Mtx{b} = \Bracks{\begin{array}{c} 12\\0\\6 \end{array}}.
\]
\begin{bksolution}
By \thmref{iterconvg}, with \Mtx{Q} the identity matrix, we have convergence if
and only if 
\[\dltwo{\Mtx{I} - \omega\Mtx{A}} < 1\]
We now use the fact that ``eigenvalues commute with polynomials;'' that is if
$f(x)$ is a polynomial and $\lambda$ is an eigenvalue of a matrix \Mtx{A}, then
$f(\lambda)$ is an eigenvalue of the matrix $f(\Mtx{A}).$  In this case the
polynomial we consider is $f(x) = x^0 - \omega x^1.$  
\ifthenelse{\boolean{hasoctave}}{Using octave or Matlab
you will find that the eigenvalues of \Mtx{A} are approximately
$7.7321,4.2679,$ and $4$.}{The eigenvalues of \Mtx{A} are approximately
$7.7321,4.2679,$ and $4$.}
Thus the eigenvalues of $\Mtx{I} - \omega\Mtx{A}$
are approximately
\[1 - 7.7321 \omega,\,1- 4.2679 \omega,\,1-4\omega.\]
With some work it can be shown that all three of these values will be less than
one in absolute value if and only if 
\[ 0 < \omega < \frac{3}{7.7321} \approx 0.388\]
(See also \chexref{richardseigen}.)

Compare this to the results of \bkexpref{richbadomega}, where for this system,
$\omega = 1$ apparently did not lead to convergence, while for
\bkexpref{richgoodomega}, with $\omega = 1/6,$ convergence was observed.

%best omega is around 0.318107283098466 ...
\end{bksolution}
\end{bkexprob}%UNFOLD

%Note that not all matrices with spectral radius less than $1$ are diagonally
%dominant.   For example:
%\[\Mtx{A} = \Bracks{\begin{array}{cc} 0 & \oneby{4}\\1 & 0\end{array}}.\]
%This matrix is not diagonally dominant, but has spectral radius of \half.

%UNFOLD
%%%%%%%%%%%%%%%%%%%%%%%%%%%%%%%%%%%%%%%%%%%%%%%
\subsection{A Free Lunch?}\label{subsec:freelunch}%FOLDUP

The analysis leading to \thmref{iterconvg} leads to an interesting possible
variant of the iterative scheme.  For simplicity we will only consider an
alteration of Richardson's Iteration.  In the altered algorithm we presuppose
the existence, via some oracle, of a sequence of weightings, $\omega_i,$ which
we use in each iterative update.  Thus our algorithm becomes:
\begin{compactenum}
\item Select some initial iterate \thex{0}.
\item Given iterate \thex{k-1}, define
\[\thex{k} = \Parens{\Mtx{I} - \omega_k \Mtx{A}} \thex{k-1} + \omega_k
\vect{b}.\]
\end{compactenum}

Following the analysis for \thmref{iterconvg}, it can be shown that
\[\thee{k} = \Parens{\Mtx{I} - \omega_k \Mtx{A}} \thee{k-1}\]
where, again, $\thee{k} = \thex{k} - x,$ with $x$ the actual solution to the
linear system.   Expanding \thee{k-1} similarly gives
\begin{align*}
\thee{k} &= 
  \Parens{\Mtx{I} - \omega_k\Mtx{A}} \Parens{\Mtx{I} � \omega_{k-1}\Mtx{A}} \thee{k-2}\\
	&=  \Parens{\prod_{i=1}^k \Mtx{I} - \omega_i \Mtx{A}} \thee{0}.
\end{align*}

Remember that we want \thee{k} to be small in magnitude, or, better yet, to be
zero.  One way to guarantee that \thee{k} is zero, regardless of the choice of
\thee{0} it to somehow ensure that the matrix
\[\Mtx{B} = {\prod_{i=1}^k \Mtx{I} - \omega_i \Mtx{A}}\]
has all zero eigenvalues.  

We again make use of the fact that eigenvalues ``commute'' with polynomials to
claim that if $\lambda_j$ is an eigenvalue of \Mtx{A}, then 
\[{\prod_{i=1}^k 1 - \omega_i \lambda_j }\]
is an eigenvalue of \Mtx{B}.  This eigenvalue is zero if one of the
$\omega_i$ for $1\le i\le k$ is $1/\lambda_j.$  This suggests how we are to
pick the weightings: let them be the inverses of the eigenvalues of \Mtx{A}.

In fact, if \Mtx{A} has a small number of distinct eigenvalues, say $m$
eigenvalues, then convergence to the exact solution could be guaranteed after
only $m$ iterations, regardless of the \emph{size} of the matrix.  

As you may have guessed from the title of this subsection, this is not exactly
a practical algorithm.  The problem is that it is not simple to find, for a
given arbitrary matrix \Mtx{A}, one, some, or all its eigenvalues.  This
problem is of sufficient complexity to outweigh any savings to be gotten from
our ``free lunch'' algorithm.  

However, in some limited situations this algorithm might be practical if the
eigenvalues of \Mtx{A} are known \apriori.

%UNFOLD
%UNFOLD
%%%%%%%%%%%%%%%%%%%%%%%%%%%%%%%%%%%%%%%%%%%%%%%
%\section{Exercises}%FOLDUP
\begin{bkexs}


\item Find the LU decomposition of the following matrices, using \naive
Gaussian Elimination
\begin{inparaenum}
\item $\threebythree3{-1}{-2}9{-1}{-4}{-6}{10}{13}$
\item $\threebythree8{24}{16}1{12}{11}4{13}{19}$
\item $\threebythree{-3}{6}{0}{1}{-2}{0}{-4}{5}{-8}$
\end{inparaenum}

\item Perform back substitution to solve the equation
\[\left[\begin{array}{cccc} 
	1 & 3 & 5 & 3\\
	0 & 2 & 4 & 3\\
	0 & 0 & 2 & 1\\
	0 & 0 & 0 & 2\end{array}\right] \vect{x} = 
\left[\begin{array}{c} 
-1 \\ -1 \\ 1 \\ 2\end{array}\right]\]

\item Perform \Naive Gaussian Elimination to prove Cramer's rule for the 2D
case. That is, prove that the solution to
\[
\Bracks{\begin{array}{cc}
a & b \\
c & d \\
\end{array}}
\Bracks{\begin{array}{c}
x \\
y \\
\end{array}}
=
\Bracks{\begin{array}{c}
f \\
g \\
\end{array}}
\]
is given by 
\[
y = \frac{\det \Bracks{\begin{array}{cc} a & f \\ c & g \end{array}}}
{\det \Bracks{\begin{array}{cc} a & b \\ c & d \end{array}}}
\quad\text{and}\quad
x = \frac{\det \Bracks{\begin{array}{cc} f & b \\ g & d \end{array}}}
{\det \Bracks{\begin{array}{cc} a & b \\ c & d \end{array}}}
\]
\label{chex:cramer}

%%%%%%%%%%%%%%%%%%%%%%%
\item Implement Cramer's rule to solve a pair of linear equations in $2$
variables.  Your m-file should have header line like:
\begin{verbatim}
function x = cramer2(A,b)
\end{verbatim}
where \texttt{A} is a $2\cross2$ matrix, and \texttt{b} and \texttt{x} are
$2\cross1$ vectors.  Your code should find the \texttt{x} such that 
$\mathtt{A x} = \mathtt{b}.$ (See \chexref{cramer})

Test your code on the following (augmented) systems:\\
	\begin{inparaenum}
	\item \,$\Parens{\begin{array}{rr|r}3 & -2 & 1\\ 4 & -3 & -1 \end{array}}$\,%
	\item \,$\Parens{\begin{array}{rr|r}1.24 &-3.48 & 1\\-0.744 &2.088 & 2
	\end{array}}$\,%
	\item \,$\Parens{\begin{array}{rr|r}1.24 &-3.48 & 1\\-0.744 &2.088 &
	-0.6\end{array}}$\,\\%
	\item \,$\Parens{\begin{array}{rr|r} -0.0590 & 0.2372 & -0.3528\\
	 0.1080 & -0.4348 & 0.6452\end{array}}$\,%
	\end{inparaenum}

%%%%%%%%%%%%%%%%%%%%%%%
\item Given two lines parametrized by $\vect{f}(t) = \vect{a} t + \vect{b},$ and
$\vect{g}(s) = \vect{c} s + \vect{d},$ set up a linear $2\cross2$ system of
equations to find the $t,s$ at the point of intersection of the two lines.
If you were going to write a program to detect the intersection of two lines,
how would you detect whether they are parallel?  How is this related to the
form of the solution to a system of two linear equations? (See
\chexref{cramer})

%%%%%%%%%%%%%%%%%%%%%%%
\item Prove that the inverse of the matrix
\[
\Bracks{\begin{array}{ccccc}
1 & 0 & 0 & \cdots & 0\\
a_{2} & 1 & 0 & \cdots & 0\\
a_{3} & 0 & 1 & \cdots & 0\\
\vdots & \vdots & \vdots & \ddots & \vdots\\
a_{n} & 0 & 0 & \cdots & 1
\end{array}}
\]
is the matrix
\[
\Bracks{\begin{array}{ccccc}
1 & 0 & 0 & \cdots & 0\\
- a_{2} & 1 & 0 & \cdots & 0\\
- a_{3} & 0 & 1 & \cdots & 0\\
\vdots & \vdots & \vdots & \ddots & \vdots\\
- a_{n} & 0 & 0 & \cdots & 1
\end{array}}
\]
\begin{bkhint}Multiply them together.\end{bkhint}
\label{chex:invM}

%%%%%%%%%%%%%%%%%%%%%%%
\item Under the strategy of scaled partial pivoting,
which row of the following matrix will be the first pivot row?
\[\left[\begin{array}{ccccc} 
	  10 &  17 & -10 & 0.1 & 0.9 \\
	 - 3 &   3 &  -3 & 0.3 & - 4 \\
	 0.3 & 0.1 & 0.01& - 1 & 0.5 \\
	   2 &   3 &   4 &  -3 &   5 \\
	  10 & 100 &   1 & 0.1 &   0 \\
  \end{array}\right]\]
%%%%%%%%%%%%%%%%%%%%%%%
\item Let \Mtx{A} be a symmetric positive definite $n\cross n$ matrix with $n$
distinct eigenvalues. Letting $\they{0} = \vect{b}/\norm{\vect{b}},$ consider the iteration
\[\they{k+1} = \frac{\Mtx{A}\they{k}}{\norm{\Mtx{A} \they{k}}}.\]
	\begin{compactenum}
	\item What is \norm{\they{k}}?
	\item Show that $\they{k} = \Mtx{A}^k \vect{b} / \norm{\Mtx{A}^k \vect{b}}.$
	\item Show that as $k\to\infty$, \they{k} converges to the (normalized) eigenvector
	associated with the largest eigenvalue of \Mtx{A}.
	\end{compactenum}

%%%%%%%%%%%%%%%%%%%%%%%
\item Consider the equation
\[\left[\begin{array}{ccc} 
	1 & 3 & 5 \\
	-2 & 2 & 4 \\
	4 & -3 & -4 \end{array}\right] \vect{x} = 
\left[\begin{array}{c} 
-5 \\ -6 \\ 10\end{array}\right]\]
Letting $\thex{0} = \trans{\Bracks{1\,\,1\,\,0}},$ find the iterate \thex{1} by one
step of Richardson's Method.  And by one step of Jacobi Iteration. And by Gauss Seidel.
%%%%%%%%%%%%%%%%%%%%%%%
\item Let \Mtx{A} be a symmetric $n\cross n$ matrix with eigenvalues in the
interval \ccinv{\alpha}{\beta}, with $0 < \alpha \le \beta,$ and
$\alpha+\beta\ne0.$
Consider Richardson's Iteration
\[\thex{k+1} = \Parens{\Mtx{I} - \omega\Mtx{A}} \thex{k} + \omega\vect{b}.\]
	Recall that $\thee{k+1} = \Parens{\Mtx{I} - \omega\Mtx{A}} \thee{k}.$
	\begin{compactenum}
	\item Show that the eigenvalues of $\Mtx{I} - \omega\Mtx{A}$ are in 
	the interval \ccinv{1-\omega\beta}{1-\omega\alpha}.  
	\item Prove that 
	\[\max\sngtn{\abs{\lambda} : 1-\omega\beta \le \lambda \le 1-\omega\alpha}\]
	is minimized when we choose $\omega$ such that $1 - \omega\beta =
	-\Parens{1-\omega\alpha}.$ 
	\begin{bkhint}It may help to look at the graph of something versus
	$\omega$.\end{bkhint}
	\item Show that this relationship is satisfied by $\omega = 2/\Parens{\alpha
	+ \beta}$.
	\item For this choice of $\omega$ show that the spectral radius of $\Mtx{I} -
	\omega\Mtx{A}$ is 
	\[\frac{\abs{\alpha - \beta}}{\abs{\alpha + \beta}}.\]
	\item Show that when $0 < \alpha,$ this quantity is always smaller than $1.$
	\item Prove that if \Mtx{A} is positive definite, then there is an $\omega$
	such that Richardson's Iteration with this $\omega$ will converge for any
	choice of \thex{0}.
	\item For which matrix do you expect faster convergence of Richardson's
	Iteration: 
	\Mtx{A_1} with eigenvalues in \ccinv{10}{20} or
	\Mtx{A_2} with eigenvalues in \ccinv{1010}{1020}? Why?
	\end{compactenum}
\label{chex:richardseigen}


%
%An easier condition to check is \emph{diagonal dominance}.   The matrix \Mtx{A}
%is said to be diagonally dominant if for $j=\onetox{n},$
%\[\abs{a_{jj}} > \sum_{i=1,i\ne j}^n \abs{a_{ji}}.\]
%Thus in a diagonally dominant matrix, a diagonal element is larger, in absolute
%value, than the sum of the absolute values of the rest of the row.  
%A diagonally dominant matrix plus the right choice of \Mtx{Q} gives the right
%eigenvalues, as the following theorem asserts:
%
%\begin{bktheorem}
%If the matrix \Mtx{A} is diagonally dominant, then the Jacobi and Gauss Seidel 
%iterative solution schemes converge for any starting \thex{0}.
%\end{bktheorem}

%%%%%%%%%%%%%%%%%%%%%%%
\item Implement Richardson's Iteration to solve the system $\Mtx{A}\vect{x} =
\vect{b}.$  Your m-file should have header line like:
\begin{verbatim}
function xk = richardsons(A,b,x0,w,k)
\end{verbatim}
Your code should return \thex{k} based on the iteration
\[\thex{j+1} = \thex{j} - \omega \Parens{\Mtx{A}\thex{j} - \vect{b}}.\]
Let \texttt{w} take the place of $\omega,$ and let \texttt{x0} be the initial
iterate \thex{0}.  Test your code for $\Mtx{A},\vect{b}$ for which you know the
actual solution to the problem. 
\begin{bkhint}Start with \Mtx{A} and the
solution \thex{} and \emph{generate} \vect{b}.\end{bkhint}
Test your code on the following matrices:
	\begin{compactitem}
	\item Let \Mtx{A} be the \emph{Hilbert Matrix}\index{Hilbert Matrix}.  

	\ifthenelse{\boolean{hasoctave}}{%FOLDUP
	This is generated by the octave command \texttt{A = hilb(10)}, or whatever
	(smallish) integer.}{}%UNFOLD
	Try different values of $\omega,$ including $\omega = 1.$
	\item Let \Mtx{A} be a \emph{Toeplitz} matrix of the form:
	\[
	\Mtx{A} = \Bracks{\begin{array}{ccccc}
	-2 & 1 & 0 & \cdots & 0\\
	1 & -2 & 1 & \cdots & 0\\
	0 & 1 & -2 & \cdots & 0\\
	\vdots & \vdots & \vdots & \ddots & \vdots\\
	0 & 0 & 0 & \cdots & -2
	\end{array}}
	\]
	\ifthenelse{\boolean{hasoctave}}{%FOLDUP
	These can be generated by the 
	octave command \texttt{A = toeplitz([-2 1 0 0 0 0 0 0])}.}{}%UNFOLD
	Try different values of $\omega,$ including $\omega = -1/2.$
	\end{compactitem}

%%%%%%%%%%%%%%%%%%%%%%%
\item Let \Mtx{A} be a nonsingular $n\cross n$ matrix.  We wish to solve
$\Mtx{A}\vect{x} = \vect{b}.$  Let \thex{0} be some
starting vector, let \dee{k} be
$\operatorname{span}\sngtn{\ther{0},\Mtx{A}\ther{0},\ldots,\Mtx{A}^{k}\ther{0}},$
and let \pee{k} be the set of polynomials, $p(x)$ of degree $k$ with $p(0) =
1.$

Consider the following iterative method:  Let \thex{k+1} be the \thex{} that solves
\[\min\limits_{\thex{} \in \thex{0} + \dee{k}} \norm{\vect{b} - \Mtx{A}\thex{}}.\]
Let $\ther{k} = \vect{b} - \Mtx{A}\thex{k}.$

	\begin{compactenum}
	\item Show that if $\thex{} \in \thex{0} + \dee{k},$ then $\vect{b} -
	\Mtx{A}\thex{} = p(\Mtx{A}) \ther{0}$ for some $p \in \pee{k}.$
	\item Prove that, conversely, for any $p \in \pee{k}$ there is some
	$\thex{} \in \thex{0} + \dee{k},$ such that $\vect{b} -
	\Mtx{A}\thex{} = p(\Mtx{A}) \ther{0}.$ 
	\item Argue that 
	\[\norm{\ther{k+1}} = \min\limits_{p \in \pee{k}} \norm{p(\Mtx{A})\ther{0}}.\]
	\item Prove that this iteration converges in at most $n$ steps.
	\begin{bkhint}Argue for the existence of a polynomial in \pee{n} that
	vanishes at all the eigenvalues of \Mtx{A}.  Use this polynomial to show that 
	$\norm{\ther{n}} \le 0$.\end{bkhint}

	\end{compactenum}

\end{bkexs}
%UNFOLD
%for vim modeline: (do not edit)
% vim:ts=2:sw=2:tw=79:fdm=marker:fmr=FOLDUP,UNFOLD:cms=%%s:tags=tags;:syntax=tex:filetype=tex:ai:si:cin:nu:fo=croqt:cino=p0t0c5(0:
