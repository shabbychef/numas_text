%%%%(c) COPYRIGHT NOTICE%FOLDUP
%%%%(c)
%%%%(c)  This file is a portion of the source for the textbook
%%%%(c)
%%%%(c)    Numerical Methods Course Notes,
%%%%(c)    Copyright 2004-2010 by Steven E. Pav
%%%%(c)
%%%%(c)  See the file COPYING.txt for copying conditions
%%%%(c)
%%%%(c)%UNFOLD

%%throat clearing%FOLDUP
\typeout{-- intro.tex}
\typeout{-- N� 2004-2010 Steven E. Pav}
%UNFOLD

%%local commands%FOLDUP
\providecommand{\bifo}[3][{}]{\BIFO{#1}{{#2},{#3}}}
%UNFOLD

\chapter{Introduction}

%%%%%%%%%%%%%%%%%%%%%%%%%%%%%%%%%%%%%%%%%%%%%%%
\section{Taylor's Theorem}%FOLDUP
\label{sec:taylors}

Recall from calculus the Taylor's series for a function, $f(x)$, expanded about some
number, $c$, is written as
\[f(x) \sim  a_0 + a_1 \Parens{x-c} + a_2 \Parens{x-c}^2 + \ldots.\]%
Here the symbol $\sim$ is used to denote a ``formal series,'' meaning that 
convergence is not guaranteed in general.  The constants $a_i$ are related to
the function $f$ and its derivatives evaluated at $c$.  When $c=0,$ this is a
MacLaurin series.

For example we have the following Taylor's series (with $c=0$):

\begin{empheq}[innerbox=\widefbox]{align}
\exp{x} &= 1 + x + \frac{x^2}{2!} + \frac{x^3}{3!} + \ldots\\
\sin\Parens{x} &= x - \frac{x^3}{3!} + \frac{x^5}{5!} - \ldots\\
\cos\Parens{x} &= 1 - \frac{x^2}{2!} + \frac{x^4}{4!} - \ldots
\end{empheq}

\begin{bktheorem}[Taylor's Theorem]\label{bkthm:taylors}%FOLDUP
If $f(x)$ has derivatives of order $0,1,2,\ldots,n+1$ on the closed interval
\clinv{a}{b}, then for any $x$ and $c$ in this interval
\[
f(x) = \sum_{k=0}^{n} \frac{f^{\Parens{k}}\Parens{c} \, \Parens{x-c}^k}{k!} +
\frac{f^{\Parens{n+1}}\Parens{\xi} \, \Parens{x-c}^{n+1}}{\Parens{n+1}!},
\]
where $\xi$ is some number between $x$ and $c,$ and $f^{k}(x)$ is the \kth{k}
derivative of $f$ at $x.$
\end{bktheorem}
\index{Taylor's Theorem}
%UNFOLD
We will use this theorem again and again in this class.  The main usage is to
approximate a function by the first few terms of its Taylor's series
expansion;  the theorem then tells us the approximation is ``as good'' as the
final term, also known as the \emph{error term}.  That is, we can make the
following manipulation:

\begin{eqnarray*}
f(x) &=& \sum_{k=0}^{n} \frac{f^{\Parens{k}}\Parens{c} \, \Parens{x-c}^k}{k!} +
\frac{f^{\Parens{n+1}}\Parens{\xi} \, \Parens{x-c}^{n+1}}{\Parens{n+1}!}\\
f(x) - \sum_{k=0}^{n} \frac{f^{\Parens{k}}\Parens{c} \, \Parens{x-c}^k}{k!} &=& 
\frac{f^{\Parens{n+1}}\Parens{\xi} \, \Parens{x-c}^{n+1}}{\Parens{n+1}!}\\
\abs{f(x) - \sum_{k=0}^{n} \frac{f^{\Parens{k}}\Parens{c} \, \Parens{x-c}^k}{k!}} &=& 
\frac{\abs{f^{\Parens{n+1}}\Parens{\xi}} \abs{x-c}^{n+1}}{\Parens{n+1}!}.
\end{eqnarray*}

On the left hand side is the difference between $f(x)$ and its approximation by
Taylor's series.  We will then use our knowledge about
$f^{\Parens{n+1}}\Parens{\xi}$ on the interval \clinv{a}{b} to find some
constant $M$ such that

\begin{eqnarray*}
\abs{f(x) - \sum_{k=0}^{n} \frac{f^{\Parens{k}}\Parens{c} \, \Parens{x-c}^k}{k!}} &=& 
\frac{\abs{f^{\Parens{n+1}}\Parens{\xi}} \abs{x-c}^{n+1}}{\Parens{n+1}!} \le M
\abs{x-c}^{n+1}.
\end{eqnarray*}

\begin{bkexprob}%FOLDUP
Find an approximation for $f(x) = \sin x,$ expanded about $c=0,$ using $n=3.$
\begin{bksolution}
Solving for $f^{\Parens{k}}$ is fairly easy for this
function.  We find that
\begin{eqnarray*}
f(x) = \sin x &=& \sin (0) + \frac{\cos(0) \, x}{1!} + \frac{-\sin (0) \, x^2}{2!} +
\frac{- \cos (0) \, x^3}{3!} + \frac{\sin (\xi) \, x^4}{4!}\\
&=& x - \frac{x^3}{6} + \frac{\sin (\xi ) \, x^4}{24},
\end{eqnarray*}
so
\begin{eqnarray*}
\abs{\sin x - \Parens{x - \frac{x^3}{6}}} &=& \abs{\frac{\sin (\xi ) \, x^4}{24}}
\le \frac{x^4}{24},
\end{eqnarray*}
because $\abs{\sin(\xi)} \le 1.$
\end{bksolution}
\end{bkexprob}%UNFOLD
\begin{bkexprob}%FOLDUP
Apply Taylor's Theorem for the case $n=1.$  
\begin{bksolution}
Taylor's Theorem for $n=1$ states:
Given a function, $f(x)$ with a continuous derivative on \clinv{a}{b}, then 
\[f(x) = f(c) + f'(\xi) (x-c)\]
for some $\xi$ between $x,c$ when $x,c$ are in \clinv{a}{b}.\\
This is the {Mean Value Theorem}.  As a one-liner, the MVT says that at some
time during a trip, your velocity is the same as your average velocity for the
trip.  
\end{bksolution}
\end{bkexprob}%UNFOLD

\begin{bkexprob}%FOLDUP
Apply Taylor's Theorem to expand $f(x) = x^3 - 21x^2 + 17$ around $c=1$.
\begin{bksolution} Simple calculus gives us
\begin{eqnarray*}
f^{\Parens{0}}(x) &=& x^3 - 21x^2 + 17,\\
f^{\Parens{1}}(x) &=& 3 x^2 - 42x,\\
f^{\Parens{2}}(x) &=& 6 x - 42,\\
f^{\Parens{3}}(x) &=& 6,\\
f^{\Parens{k}}(x) &=& 0.
\end{eqnarray*}
with the last holding for $k>3.$  Evaluating these at $c=1$ gives
\[
f(x) = -3 + -39 (x-1) + \frac{-36 \Parens{x-1}^2}{2} + \frac{6
\Parens{x-1}^3}{6}.
\]
Note there is no error term, since the higher order derivatives are identically
zero.  By carrying out simple algebra, you will find that the above expansion
is, in fact, the function $f(x).$
\end{bksolution}
\end{bkexprob}%UNFOLD

There is an alternative form of Taylor's Theorem, in this case substituting
$x+h$ for $x,$ and $x$ for $c$ in the more general version.  This gives

\begin{bktheorem}[Taylor's Theorem, Alternative Form]%FOLDUP
If $f(x)$ has derivatives of order $0, 1, \ldots, n+1$ on the closed interval
\clinv{a}{b}, then for any $x$ in this interval and any $h$ such that $x+h$ is
in this interval,
\[
f(x+h) = \sum_{k=0}^{n} \frac{f^{\Parens{k}}\Parens{x} \, \Parens{h}^k}{k!} +
\frac{f^{\Parens{n+1}}\Parens{\xi} \, \Parens{h}^{n+1}}{\Parens{n+1}!},
\]
where $\xi$ is some number between $x$ and $x+h.$
\end{bktheorem}%UNFOLD
\index{Taylor's Theorem}

We generally apply this form of the theorem with $h \rightarrow 0.$  This leads
to a discussion on the matter of \emph{Orders of Convergence}.  The following
definition will suffice for this class

\begin{bkdefinition}%FOLDUP
We say that a function $f(h)$ is in the class \bigo{h^k} (pronounced ``big-Oh
of $h^k$'') if there is some
constant $C$ such that
\[\abs{f(h)} \le C \abs{h}^k\]
for all $h$ ``sufficiently small,'' \ie smaller than some $h^*$ in absolute
value.

For a function $f \in \bigo{h^k}$ we sometimes write $f = \bigo{h^k}.$  We
sometimes also write \bigo{h^k}, meaning some function which is a member of this class.
\end{bkdefinition}
%UNFOLD
\index{big-O}
%
%
Roughly speaking, through use of the ``Big-O'' function we can write an
expression without ``sweating the small stuff.''  This can give us an intuitive
understanding of how an approximation works, without losing too many of the
details.

\begin{bkexample}\label{bkex:lnexp}%FOLDUP
Consider the Taylor expansion of $\ln x$:
\begin{eqnarray*}
\ln \Parens{x + h} &=& \ln x + \frac{\Parens{1/x} \, h}{1} + \frac{\Parens{-1/x^2} \, h^2}{2} +
\frac{\Parens{2/\xi^3} \, h^3}{6}
\end{eqnarray*}
Letting $x=1,$ we have
\begin{eqnarray*}
\ln \Parens{1 + h} &=& h - \half[h^2] + \oneby{3 \xi^3} h^3.
\end{eqnarray*}
Using the fact that $\xi$ is between $1$ and $1+h,$ as long as $h$ is
relatively small (say smaller than \half), the term $\oneby{3 \xi^3}$ can be
bounded by a constant, and thus
\begin{eqnarray*}
\ln \Parens{1 + h} &=& h - \half[h^2] + \bigo{h^3}.
\end{eqnarray*}

Thus we say that $h - \half[h^2]$ is a \bigo{h^3} approximation to $\ln (1+h).$
For example
\[
\ln(1+0.01) \approx 0.009950331 \approx 0.00995 = 0.01 - \half[0.01^2]. 
\]

\end{bkexample}%UNFOLD

%\subsection{Alternating Series Theorem}%FOLDUP
%Another theorem for proving convergence is the following:
%
%\begin{bktheorem}[Alternating Series Theorem]%FOLDUP
%If $a_0 \ge a_1 \ge a_2 \ge a_3 \ge \ldots \ge 0,$ and $\lim_{k\rightarrow\infty} a_k =
%0$ then the series 
%\[
%\sum_{k=0}^\infty \Parens{-1}^k a_k = a_0 - a_1 + a_2 - \ldots
%\]
%converges.  Moreover, if you let $S$ be the value of the sum, and $S_n$ be the
%partial sum $\sum_{k=0}^n \Parens{-1}^k a_k$, then 
%\[
%\abs{S_n - S} \le a_{n+1}.
%\]
%\end{bktheorem}%UNFOLD
%
%This theorem can help us figure out how many terms to take in an approximation.
%
%\begin{bkexample}%FOLDUP
%Use Taylor's Theorem on $f(x) = \ln\Parens{1+x},$ then use the AST to figure
%out how many terms are needed to get an accurate approximation for $\ln2.$
%\\\emph{Solution:}  The expansion for this function is actually
%\begin{eqnarray*}
%\ln\Parens{1+x} &=& x - \frac{x^2}{2} + \frac{x^3}{3} - \frac{x^4}{4} + \ldots
%= \sum_{k=1}^\infty \frac{\Parens{-1}^{k-1} x^k}{k}.
%\end{eqnarray*}
%Thus
%\begin{eqnarray*}
%\ln2 = \ln\Parens{1+x} &=& 1 - \oneby{2} + \oneby{3} - \oneby{4} + \ldots
%\end{eqnarray*}
%This is an alternating series with terms
%$1,\oneby{2},\oneby{3},\oneby{4},\ldots$  How many terms are sufficient?
%Suppose we wish to find an approximation of $\ln2$ good to $9$ decimal places.
%Thus we want an error smaller than \tenex{\half}{-9}.  The AST tells us that
%\[\abs{S_n - n} \le a_{n+1} = \oneby{n+1}.\]
%Thus it suffices to take $\oneby{n+1} \le \tenex{\half}{-9}.$  That is, the AST
%tells us we need to take 2 billion terms in our approximation;  this is poor.
%\end{bkexample}%UNFOLD
%%UNFOLD
%UNFOLD
%%%%%%%%%%%%%%%%%%%%%%%%%%%%%%%%%%%%%%%%%%%%%%%
\section{Loss of Significance}%FOLDUP

Generally speaking, a computer stores a number $x$ as a mantissa and exponent,
that is $x=\tenex{\pm r}{k},$ where $r$ is a rational number of a given number of
digits in \coinv{0.1}{1}, and $k$ is an integer in a certain range.

The number of significant digits in $r$ is usually determined by the user's
input.  Operations on numbers stored in this way follow a ``lowest common
denominator'' type of rule, \ie precision cannot be gained but can be lost.
Thus for example if you add the two quantities $0.171717$ and $0.51$, then the
result should only have two significant digits;  the precision of the first
measurement is lost in the uncertainty of the second.  

This is as it should be.  However, a loss of significance can be incurred if
two nearly equal quantities are subtracted from one another.  Thus if I were to
direct my computer to subtract $0.177241$ from $0.177589,$ the result would be
$\tenex{.348}{-3},$ and three significant digits have been lost.  This loss is
called \emph{subtractive cancellation}, \index{subtractive cancellation}
and can often be avoided by rewriting the expression.  This will be made
clearer by the examples below.

Errors can also occur when quantities of radically different magnitudes are
summed. For example $0.1234 + \tenex{5.6789}{-20}$ might be rounded to $0.1234$
by a system that keeps only 16 significant digits.  This may lead to
unexpected results.

The usual strategies for rewriting subtractive expressions are completing the
square, factoring, or using the Taylor expansions, as the following examples
illustrate.  

\begin{bkexprob}%FOLDUP
Consider the stability of $\sqrt{x+1} - 1$ when $x$ is near $0.$  Rewrite the
expression to rid it of subtractive cancellation.
\begin{bksolution}Suppose that $x = \tenex{1.2345678}{-5}.$  Then $\sqrt{x+1}
\approx 1.000006173$.  If your computer (or calculator) can only keep 8
significant digits, this will be rounded to $1.0000062.$  When $1$ is
subtracted, the result is \tenex{6.2}{-6}.  Thus $6$ significant digits have
been lost from the original.

To fix this, we \emph{rationalize} the expression
\begin{eqnarray*}
\sqrt{x+1} - 1 &=& \sqrt{x+1} - 1 \frac{\sqrt{x+1} + 1}{\sqrt{x+1} + 1} 
=	\frac{x + 1 - 1}{\sqrt{x+1} + 1} 
=	\frac{x}{\sqrt{x+1} + 1}.
\end{eqnarray*}
This expression has no subtractions, and so is not subject to subtractive
cancelling.  When $x=\tenex{1.2345678}{-5},$ this expression evaluates
approximately as 
\[\frac{\tenex{1.2345678}{-5}}{2.0000062} \approx \tenex{6.17281995}{-6}\]
on a machine with 8 digits, there is no loss of precision.
\end{bksolution}
\end{bkexprob}%UNFOLD

Note that nearly all modern computers and calculators store intermediate
results of calculations in higher precision formats.  This minimizes, but does
not eliminate, problems like those of the previous example problem.

\begin{bkexprob}\label{bkexp:quadratic}%FOLDUP
Write stable code to find the roots of the equation $x^2 + bx + c = 0.$
\begin{bksolution}The usual quadratic formula is
\[x_{\pm} = \frac{-b \pm \sqrt{b^2 - 4c}}{2}\]
Supposing that $b \gg c > 0$, the expression in the square root might be rounded to
$b^2,$ giving two roots $x_{+} = 0,\,x_{-} = -b.$  The latter root is nearly
correct, while the former has no correct digits.  To correct this problem,
multiply the numerator and denominator of $x_{+}$ by $-b - \sqrt{b^2 - 4c}$
to get
\[x_{+} = \frac{2c}{-b - \sqrt{b^2 - 4c}}\]
Now if $b \gg c > 0,$ this expression gives root $x_{+} = -c/b,$ which
is nearly correct.  This leads to the pair:
\[x_{-} = \frac{-b - \sqrt{b^2 - 4c}}{2},\quad
x_{+} = \frac{2c}{-b - \sqrt{b^2 - 4c}}\]
Note that the two roots are nearly reciprocals, and if $x_{-}$ is computed,
$x_{+}$ can easily be computed with little additional work.
\end{bksolution}
\end{bkexprob}%UNFOLD
\begin{bkexprob}%FOLDUP
Rewrite $\exp{x} - \cos x$ to be stable when $x$ is near $0.$
\begin{bksolution}Look at the Taylor's Series expansion for these
functions:
\begin{eqnarray*}
\exp{x} - \cos{x} &=& \Bracks{1 + x + \frac{x^2}{2!} + \frac{x^3}{3!} +
\frac{x^4}{4!} + \frac{x^5}{5!} + \ldots} - 
\Bracks{1 - \frac{x^2}{2!} + \frac{x^4}{4!} - \frac{x^6}{6!} + \ldots}\\
&=& x + x^2 + \frac{x^3}{3!} + \bigo{x^5}
\end{eqnarray*}
This expression has no subtractions, and so is not subject to subtractive
cancelling.  Note that we propose calculating $x + x^2 + x^3/6$ as an
\emph{approximation} of $\exp{x} - \cos{x},$ which we cannot calculate exactly
anyway.  Since we assume $x$ is nearly zero, the approximate should be
good.  If $x$ is \emph{very} close to zero, we may only have to take the first
one or two terms.  If $x$ is not so close to zero, we may need to take all
three terms, or even more terms of the expansion; if $x$ is far from zero we 
should use some other technique.
\end{bksolution}
\end{bkexprob}%UNFOLD
%UNFOLD
%%%%%%%%%%%%%%%%%%%%%%%%%%%%%%%%%%%%%%%%%%%%%%%
\section{Vector Spaces, Inner Products, Norms}%FOLDUP

We explore some of the basics of functional analysis which may be useful in
this text.

%%%%%%%%%%%%%%%%%%%%%%%%%%%%%%%%%%%%%%%%%%%%%%%
\subsection{Vector Space}%FOLDUP
\label{subsec:vectorspace}

A vector space is a collection of objects together with a binary operator which is
defined over an algebraic field.\footnote{For the purposes of this text, this algebraic field will
\emph{always} be the real field, \reals{}, though in the real world, the
complex field, \comps{}, has some currency.} 
The binary operator allows
transparent algebraic manipulation of vectors.

\begin{bkdefinition}\label{bkdef:vecspace}%FOLDUP
A collection of vectors, \vecs, with a binary addition operator, $+,$
defined over \vecs, and a scalar multiply over the real field \reals{},
forms a \emph{vector space} if

\begin{compactenum}
\item For each $\vect{u},\vect{v}\in\vecs,$ the sum $\vect{u}+\vect{v}$ is a vector 
in \vecs.
(\ie the space is ``closed under addition.'')
\item Addition is commutative: $\vect{u}+\vect{v} = \vect{v}+\vect{u}$ for each
$\vect{u},\vect{v}\in\vecs$.

\item For each $\vect{u}\in\vecs,$ and each scalar $\alpha\in\reals{}$ the scalar
product $\alpha\vect{u}$ is a vector in \vecs.
(\ie the space is ``closed under scalar multiplication.'')
\item There is a zero vector $\vect{0}\in\vecs$ such that for any
$\vect{u}\in\vecs,$ $0 \vect{u} = \vect{0},$ where $0$ is the zero of \reals{}.
\item For any $\vect{u}\in\vecs,$ $1 \vect{u} = \vect{u},$ where $1$ is the
multiplicative identity of \reals{}.
\item 
For any $\vect{u},\vect{v}\in\vecs,$ and scalars $\alpha,\beta\in\reals{},$
both
$(\alpha +_{\reals{}} \beta)\vect{u} = \alpha\vect{u} + \beta\vect{u}$
and
$\alpha\Parens{\vect{u}+\vect{v}} = \alpha\vect{u} + \alpha\vect{v}$ hold,
where $+_{\reals{}}$ is addition
in \reals{}.
(\ie scalar multiplication distributes in both ways.)

\end{compactenum}
\end{bkdefinition}
%UNFOLD
\index{vector space}

\begin{bkexample}\label{bkex:realsn}%FOLDUP
The most familiar example of a vector space is \reals{n}, which is the
collection of $n$-tuples of real numbers.  That is $\vect{u}\in\reals{n}$ is of
the form \smooshvec{u_1,u_2,\ldots,u_n}, where $u_i\in\reals{}$ for
$i=\onetox{n}$. Addition and scalar multiplication over \reals{} are defined pairwise:
\begin{align*}
 \smooshvec{u_1,u_2,\ldots,u_n} + 
 \smooshvec{v_1,v_2,\ldots,v_n} &=
 \smooshvec{u_1 + v_1,u_2 + v_2,\ldots,u_n + v_n},\quad\text{and}\\
 \alpha\smooshvec{u_1,u_2,\ldots,u_n} &=
 \smooshvec{\alpha u_1,\alpha u_2,\ldots,\alpha u_n}
\end{align*}
\end{bkexample}%UNFOLD
Note that some authors distinguish between \emph{points} in $n$-dimensional
space and \emph{vectors} in $n$-dimensional space.  We will use \reals{n} to
refer to both of them, as in this text there is no need to distinguish them
symbolically.
%\begin{bkexample}\label{bkex:compsn}%FOLDUP
%Similarly defined is \comps{n}, which is the
%collection of $n$-tuples of complex numbers.   This is a vector space over
%\comps{}.
%\end{bkexample}%UNFOLD
\begin{bkexample}\label{bkex:funcspace}%FOLDUP
Let $X\subseteq\reals{k}$ be an closed, bounded set, and let $H$ be the collection of 
all functions from $X$
to \reals{}.  Then $H$ forms a vector space under the ``pointwise'' defined
addition and scalar multiplication over \reals{}.
That is, for $u,v\in H,$ $u+v$ is the function in $H$ defined by 
\( \Bracks{u+v}\Parens{x} = u(x) + v(x)\)
for all $x\in X$.  And for $u\in H,\alpha \in \reals{},$ $\alpha u$ is the function in
\( \Bracks{\alpha u}\Parens{x} = \alpha\Parens{u(x)}\).
\end{bkexample}%UNFOLD
\begin{bkexample}\label{bkex:funcspacezero}%FOLDUP
Let $X\subseteq\reals{k}$ be a closed, bounded set, and let $H_0$ be the collection of 
all functions from $X$ to \reals{} that take the value zero on \bord{X}. 
Then $H$ forms a vector space under the ``pointwise'' defined
addition and scalar multiplication over \reals{}.
The only difference between proving $H_0$ is a vector space and the proof
required for the previous example is in showing that $H_0$ is
indeed closed under addition and scalar multiplication.  This is simple because
if $x\in\bord{X},$ then $\Bracks{u+v}(x) = u(x) + v(x) = 0 + 0,$ and thus $u+v$
has the property that it takes value $0$ on \bord{X}.  Similarly for $\alpha
u$. This would not have worked if the functions of $H_0$ were required to take
some other value on \bord{X}, like, say, $2$ instead of $0$.
\end{bkexample}%UNFOLD
\begin{bkexample}\label{bkex:polyspaces}%FOLDUP
Let \polys{n} be the collection of all `formal' polynomials of degree less 
than or equal to $n$ with coefficients from \reals{}.  Then \polys{n} forms a
vector space over \reals{}.
\end{bkexample}%UNFOLD
\begin{bkexample}\label{bkex:realmatsn}%FOLDUP
The collection of all real-valued $m\cross n$ matrices forms a vector space
over the reals with the usual scalar multiplication and matrix addition.
This space is denoted as \reals{m\cross n}.  Another way of viewing this space
is it is the space of linear functions which carry vectors of \reals{n} to
vectors of \reals{m}.
\end{bkexample}%UNFOLD
%UNFOLD
%%%%%%%%%%%%%%%%%%%%%%%%%%%%%%%%%%%%%%%%%%%%%%%
\subsection{Inner Products}%FOLDUP
\label{subsec:innerprod}

An inner product is a way of ``multiplying'' two vectors from a vector space
together to get a scalar from the same field the space is defined over (\eg a
real or a complex number).  The inner product should have the following
properties:

%\begin{bkdefinition}\label{bkdef:billinearform}%FOLDUP
%For a vector space, \vecs, defined over \fld, a binary function,
%\BIFO{}{,}, which takes two vectors of \vecs to \fld is a 
%\emph{bilinear form} if 
%
%\begin{compactenum}
%\item it is linear in both its arguments:
%\end{compactenum}
%\end{bkdefinition}
%%UNFOLD
\begin{bkdefinition}\label{bkdef:innerpro}%FOLDUP
For a vector space, \vecs, defined over \reals{}, a binary function,
\bifo{}{}, which takes two vectors of \vecs to \reals{} is an 
\emph{inner product} if 
\begin{compactenum}
\item It is symmetric: $\bifo{\vect{v}}{\vect{u}} = \bifo{\vect{u}}{\vect{v}}$.
\item It is linear in both its arguments:
\begin{align*}
\bifo{\alpha \vect{u} + \beta \vect{v}}{\vect{w}} &= 
\alpha\bifo{\vect{u}}{\vect{w}} + \beta\bifo{\vect{v}}{\vect{w}}\quad\text{and}\\
\bifo{\vect{u}}{\alpha \vect{v} + \beta \vect{w}} &= 
\alpha\bifo{\vect{u}}{\vect{v}} + \beta\bifo{\vect{u}}{\vect{w}}.
\end{align*}
A binary function for which this holds is sometimes called a \emph{bilinear
form}.
\item It is positive: $\bifo{\vect{v}}{\vect{v}} \ge 0,$ with equality holding
if and only if \vect{v} is the zero vector of \vecs.
\end{compactenum}
\end{bkdefinition}
%UNFOLD
\index{inner product}

\begin{bkexample}\label{bkex:realsnip}%FOLDUP
The most familiar example of an inner product is the \Ltwo[] (pronounced ``L
two'') inner product on the vector space \reals{n}. 
If $\vect{u} = \smooshvec{u_1,u_2,\ldots,u_n},$
and $\vect{v} = \smooshvec{v_1,v_2,\ldots,v_n},$ then letting
\[\bifo[2]{\vect{u}}{\vect{v}} = \sum_i u_i v_i\]
gives an inner product.  This inner product is the usual vector calculus dot
product and is sometimes written as \(\vect{u}\cdot\vect{v}\) or
\(\trans{\vect{u}}\vect{v}\).
%This is a specific case ($p=2$) of the more general \Lnum[]{p} inner product defined as
%\[\bifo[p]{\vect{u}}{\vect{v}} = \Parens{\sum_i \abs{u_i v_i}^{p/2}}^{2/p}\]
%no!
\end{bkexample}%UNFOLD
%\begin{bkexample}\label{bkex:compsn}%FOLDUP
%Similarly defined is \comps{n}, which is the
%collection of $n$-tuples of complex numbers.   This is a vector space over
%\comps{}.
%\end{bkexample}%UNFOLD
\begin{bkexample}\label{bkex:funcspaceip}%FOLDUP
Let $H$ be the vector space of functions from $X$ to \reals{} from
\bkexref{funcspace}.  Then for $u,v\in H$, letting
\[\bifo[H]{u}{v} = \int_X u(x) v(x) \dx,\]
gives an inner product.  This inner product is like the ``limit case'' of the
\Ltwo[] inner product on \reals{n} as $n$ goes to infinity.
\end{bkexample}%UNFOLD
%UNFOLD
%%%%%%%%%%%%%%%%%%%%%%%%%%%%%%%%%%%%%%%%%%%%%%%
\subsection{Norms}%FOLDUP
\label{subsec:norms}

A norm is a way of measuring the ``length'' of a vector:
\begin{bkdefinition}\label{bkdef:norm}%FOLDUP
A function $\dabs{\cdot}$ from a vector space, \vecs, to \reals{+} is called a 
\emph{norm} if 
\begin{compactenum}
\item It obeys the triangle inequality:
$\dabs{\vect{x}+\vect{y}} \le \dabs{\vect{x}} + \dabs{\vect{y}}.$
\item It scales positively:
$\dabs{\alpha \vect{x}} = \abs{\alpha} \dabs{\vect{x}},$ for scalar $\alpha.$
\item It is positive: $\dabs{\vect{x}} \ge 0,$ with equality only holding when
$\vect{x}$ is the zero vector.
\end{compactenum}
\end{bkdefinition}
%UNFOLD
\index{norm}

The easiest way of constructing a norm is on top of an inner product.  If 
\bifo{}{} is an inner product on vector space \vecs, then letting 
\[\dabs{\vect{u}} = \sqrt{\bifo{\vect{u}}{\vect{u}}}\] 
gives a norm on \vecs.  This is how we construct our most common norms:
\begin{bkexample}\label{bkex:realsnvecnorm}%FOLDUP
For vector $\vect{x} \in \reals{n}$, its \Ltwo[] norm is defined
\[\dltwo{\vect{x}} = \Parens{\sum_{i=1}^n x_i^2}^{\half} =
\Parens{\trans{\vect{x}}\vect{x}}^{\half}.\]
This is constructed on top of the \Ltwo[] inner product.
\end{bkexample}%UNFOLD
\begin{bkexample}\label{bkex:realsnLpnorm}%FOLDUP
The \Lnum[]{p} norm on \reals{n} generalizes the \Ltwo[] norm, and 
is defined, for $p > 0,$ as 
\[\dNormL{p}{\vect{x}} = \Parens{\sum_{i=1}^n \abs{x_i}^{p}}^{1/p}.\]
\end{bkexample}%UNFOLD
\begin{bkexample}\label{bkex:realsnLinfnorm}%FOLDUP
The \Linf[] norm on \reals{n} is defined as 
\[\dlinf{\vect{x}} = \max_{i} \abs{x_i}.\]
The \Linf[] norm is somehow the ``limit'' of the \Lnum[]{p} norm as
$p\to\infty$.
\end{bkexample}%UNFOLD
%UNFOLD


%UNFOLD
%%%%%%%%%%%%%%%%%%%%%%%%%%%%%%%%%%%%%%%%%%%%%%%
\section{Eigenvalues}%FOLDUP
\label{sec:eiegenvalues}

It is assumed the reader has some familiarity with linear algebra.  We review
the topic of eigenvalues.

\begin{bkdefinition}\label{bkdef:eigenthings}%FOLDUP
A nonzero vector \vect{x} is an \emph{eigenvector} of a given matrix \Mtx{A}, with
corresponding \emph{eigenvalue} $\lambda$ if
\[\Mtx{A} \vect{x} = \lambda \vect{x}\]
Subtracting the right hand side from the left and gathering terms gives
\[\Parens{\Mtx{A} - \lambda \Mtx{I}} \vect{x} = \vect{0}.\]
Since \vect{x} is assumed to be nonzero, the matrix 
${\Mtx{A} - \lambda \Mtx{I}}$ must be singular.  A matrix is singular if
and only if its determinant is zero.  These steps are reversible, thus we claim
$\lambda$ is an eigenvalue if and only if 
\[\det\Parens{\Mtx{A} - \lambda \Mtx{I}} = 0.\]
The left hand side can be expanded to a polynomial in $\lambda,$ of degree $n$
where $\Mtx{A}$ is an $n\cross n$ matrix.  This gives the so-called
\emph{characteristic equation}.  Sometimes eigenvectors,-values are called
characteristic vectors,-values.
\end{bkdefinition}
%UNFOLD
\index{eigenvalue}
\index{eigenvector}
%
\begin{bkexprob}\label{bkexp:findeig}%FOLDUP
Find the eigenvalues of 
\[\wrapBracks{\begin{array}{cc} 1 & 1\\4 & -2\end{array}}\]
\begin{bksolution}The eigenvalues are roots of 
\[0 = \det\wrapBracks{\begin{array}{cc} 1-\lambda & 1\\4 & -2-\lambda\end{array}} =
\Parens{1-\lambda}\Parens{-2-\lambda} - 4 = \lambda^2 + \lambda -6.
\]
This equation has roots $\lambda_1 = -3, \lambda_2 = 2.$
\end{bksolution}
\end{bkexprob}%UNFOLD

\begin{bkexprob}\label{bkexp:eigsq}%FOLDUP
Find the eigenvalues of $\Mtx{A}^2$.
\begin{bksolution}Let $\lambda$ be an eigenvalue of $\Mtx{A},$ with
corresponding eigenvector \vect{x}.  Then
\[\Mtx{A}^2 \vect{x} = \Mtx{A} \Parens{\Mtx{A} \vect{x}} = \Mtx{A}
\Parens{\lambda \vect{x}} = \lambda \Mtx{A} \vect{x} = \lambda^2 \vect{x}.\]
\end{bksolution}
\end{bkexprob}%UNFOLD

The eigenvalues of a matrix tell us, roughly, how the linear transform scales a
given matrix; the eigenvectors tell us which directions are ``purely scaled.''
This will make more sense when we talk about norms of vectors and matrices.

%%%%%%%%%%%%%%%%%%%%%%%%%%%%%%%%%%%%%%%%%%%%%%%
\subsection{Matrix Norms}%FOLDUP
\label{subsec:matrixnorms}

Given a norm \dabs{\cdot} on the vector space, \reals{n}, we can define the matrix 
norm ``subordinate'' to it, as follows:

\begin{bkdefinition}\label{bkdef:mtxnorm}%FOLDUP
Given a norm \dabs{\cdot} on \reals{n}, we define the subordinate matrix
norm on \reals{n\cross n} by
\[\dabs{\Mtx{A}} = \max_{\vect{x} \ne \vect{0}} \frac{\dabs{\Mtx{A}\vect{x}}}{\dabs{\vect{x}}}.\]
\end{bkdefinition}
%UNFOLD
\index{norm!subordinate}
We will use the subordinate two-norm for matrices.  From the definition of the
subordinate norm as a max, we 
conclude that if $\vect{x}$ is a nonzero vector then
\begin{eqnarray*}
\frac{\dltwo{\Mtx{A}\vect{x}}}{\dltwo{\vect{x}}} &\le&
\dltwo{\Mtx{A}}\quad\text{thus,}\\
\dltwo{\Mtx{A}\vect{x}} &\le& \dltwo{\Mtx{A}} \dltwo{\vect{x}}.
\end{eqnarray*}

%\begin{bkexprob}\label{bkexp:maxeig}%FOLDUP
%Let $\Lambda$ be the set of eigenvalues of $\Mtx{A}.$ Prove that
%\[
%\dltwo{\Mtx{A}} = \max\limits_{\lambda\in\Lambda} \abs{\lambda}.
%\]
%\begin{bksolution}
%First let $\lambda_M$ be an eigenvalue such that $\abs{\lambda_M} =
%\max_{\lambda\in\Lambda}\abs{\lambda}.$  Let \vect{x} be a corresponding
%eigenvector.  Then 
%\[
%\frac{\dltwo{\Mtx{A}\vect{x}}}{\dltwo{\vect{x}}} = 
%\frac{\dltwo{\lambda_M\vect{x}}}{\dltwo{\vect{x}}} = 
%\frac{\abs{\lambda_M} \dltwo{\vect{x}}}{\dltwo{\vect{x}}} = 
%\abs{\lambda_M}.
%\]
%This proves that $\dltwo{\Mtx{A}} \ge \abs{\lambda_M}.$
%
%Now for arbitrary \vect{v}, we decompose it as the linear combination of
%eigenvectors:
%$\vect{v} = \sum_i \alpha_i \vect{x_i}$
%Then
%\[\Mtx{A}\vect{v} = \Mtx{A} \sum_i \alpha_i \vect{x_i} =
% \sum_i \alpha_i \Mtx{A} \vect{x_i} =
% \sum_i \alpha_i \lambda_i \vect{x_i}.\]
%Then
%\[
%\dltwo{\Mtx{A}\vect{v}}^2 = 
%\trans{\Parens{\sum_i \alpha_i \lambda_i \vect{x_i}}} 
%{\Parens{\sum_i \alpha_i \lambda_i \vect{x_i}}} 
%= \sum_i\sum_j \alpha_i \alpha_j \lambda_i\lambda_j \trans{\vect{x_i}}\vect{x_j}
%\le \lambda_M^2 \sum_i\sum_j \alpha_i \alpha_j \trans{\vect{x_i}}\vect{x_j}
%= \lambda_M^2 \dltwo{\vect{v}}^2.
%\]
%Thus $\dltwo{\Mtx{A}\vect{v}}\le\abs{\lambda_M}\dltwo{\vect{v}},$ and thus
%$\dltwo{\Mtx{A}} \le \abs{\lambda_M}.$
%\end{bksolution}
%\end{bkexprob}%UNFOLD
%The previous proof is a bit more involved than I would expect you to follow.  I
%include it for completeness.
\begin{bkexample}\label{bkexp:maxeig}%FOLDUP
Strange but true:  If $\Lambda$ is the set of eigenvalues of $\Mtx{A},$ then
\[
\dltwo{\Mtx{A}} = \max\limits_{\lambda\in\Lambda} \abs{\lambda}.
\]
\end{bkexample}%UNFOLD
\begin{bkexprob}\label{bkexp:mnormCS}%FOLDUP
Prove that 
\[
\dltwo{\Mtx{A}\Mtx{B}} \le \dltwo{\Mtx{A}} \dltwo{\Mtx{B}}.
\]
\begin{bksolution}
\[\dltwo{\Mtx{A}\Mtx{B}} \defeq 
\max_{\vect{x} \ne \vect{0}} \frac{\dltwo{\Mtx{A}\Mtx{B}\vect{x}}}{\dltwo{\vect{x}}}
\le 
\max_{\vect{x} \ne \vect{0}} \frac{\dltwo{\Mtx{A}}\dltwo{\Mtx{B}\vect{x}}}{\dltwo{\vect{x}}}
=
{\dltwo{\Mtx{A}}\dltwo{\Mtx{B}}}.
\]
\end{bksolution}
\end{bkexprob}%UNFOLD
%UNFOLD

%UNFOLD
%%%%%%%%%%%%%%%%%%%%%%%%%%%%%%%%%%%%%%%%%%%%%%%
%\section{Exercises}%FOLDUP
\begin{bkexs}
\item Suppose $f\in\bigo{h^k}.$ Show that $f\in\bigo{h^m}$ for any $m$ with $0
< m < k.$ \begin{bkhint}Take $h^* < 1.$\end{bkhint}
Note this may appear counterintuitive, unless you remember that \bigo{h^k}
is a \emph{better} approximation than \bigo{h^m} when $m < k$.
\item Suppose $f\in\bigo{h^k},$ and $g\in\bigo{h^m}.$  Show that
$fg \in \bigo{h^{k+m}}.$
\item Suppose $f\in\bigo{h^k},$ and $g\in\bigo{h^m},$ with $m < k.$ Show that
$f + g \in \bigo{h^{m}}.$
\item Prove that $f(h) = -3 h^5$ is in \bigo{h^5}.
\item Prove that $f(h) = h^2 + 5h^{17}$ is in \bigo{h^2}.
\item Prove that $f(h) = h^3$ is \emph{not} in \bigo{h^4} \begin{bkhint}Proof
by contradiction.\end{bkhint}
\item Prove that $\sin(h)$ is in \bigo{h}.

\item Find a \bigo{h^3} approximation to $\sin{h}.$
\item Find a \bigo{h^4} approximation to $\ln(1+h).$  Compare the approximate
value to the actual when $h=0.1$.  How does this approximation compare to the
\bigo{h^3} approximate from \bkexref{lnexp} for $h=0.1$?

\item Suppose that $f \in \bigo{h^k}$.  Can you show that $f' \in
\bigo{h^{k-1}}$?
\item Rewrite $\sqrt{x+1} - \sqrt{1}$ to get rid of subtractive cancellation
when $x\approx0$.
\item Rewrite $\sqrt{x+1} - \sqrt{x}$ to get rid of subtractive cancellation
when $x$ is very large.

\item Use a Taylor's expansion to rid the expression $1 - \cos x$ of
subtractive cancellation for $x$ small.  Use a \bigo{x^5} approximate.
\item Use a Taylor's expansion to rid the expression $1 - \cos^2 x$ of
subtractive cancellation for $x$ small.  Use a \bigo{x^6} approximate.
\item Calculate $\cos(\pi/2 + 0.001)$ to within $8$
decimal places by using the Taylor's expansion.

\item Prove that if \vect{x} is an eigenvector of $\Mtx{A}$ then
$\alpha \vect{x}$ is also an eigenvector of $\Mtx{A},$ for the same eigenvalue.
Here $\alpha$ is a nonzero real number.
\item Prove, by induction, that if $\lambda$ is an eigenvalue of $\Mtx{A}$ then
$\lambda^k$ is an eigenvalue of $\Mtx{A}^k$ for integer $k>1.$
The base case was done in \bkexpref{eigsq}.
%\item Prove that if $\lambda_a$ is an eigenvalue of \Mtx{A}, and $\lambda_b$
%is an eigenvalue of \Mtx{B} \emph{for the same eigenvector}, then $\lambda_a + \lambda_b$ is an eigenvalue of
%the matrix $\Mtx{A} + \Mtx{B}.$
\item Let $\Mtx{B} = \sum_{i=0}^k \alpha_i \Mtx{A}^i,$ where $\Mtx{A}^0 =
\Mtx{I}.$  Prove that if $\lambda$ is an eigenvalue of \Mtx{A}, then 
$\sum_{i=0}^k \alpha_i \lambda^i$ is an eigenvalue of \Mtx{B}.
Thus for polynomial $p(x)$, $p(\lambda)$ is an eigenvalue of $p(\Mtx{A}).$
\item Suppose \Mtx{A} is an invertible matrix with eigenvalue $\lambda.$  Prove
that $\lambda^{-1}$ is an eigenvalue for $\invs{\Mtx{A}}.$

\item Suppose that the eigenvalues of \Mtx{A} are $1,10,100.$  Give the
eigenvalues of $\Mtx{B} = 3\Mtx{A}^3 - 4\Mtx{A}^2 + \Mtx{I}.$ Show that \Mtx{B}
is singular.

\item Show that if $\dltwo{\vect{x}}=r,$ then \vect{x} is on a sphere centered at the origin of radius $r,$ in \reals{n}.
\item If $\norm{\vect{x}} = 0,$ what does this say about vector \vect{x}?
\item Letting $\vect{x} = \trans{\Bracks{3\,\,4\,\,12}},$ what is \dltwo{\vect{x}}?
\item What is the norm of 
\[\Mtx{A} = 
  \left[\begin{array}{ccccc} 
	 1 & 0 & 0 & \cdots & 0 \\
	 0 & 1/2 & 0 & \cdots & 0 \\
	 0 & 0 & 1/3 & \cdots & 0 \\
	 \vdots & \vdots & \vdots & \ddots & \vdots \\
	 0 & 0 & 0 & \cdots & 1/n
  \end{array}\right]?\]



\item Show that $\dltwo{\Mtx{A}} = 0$ implies that \Mtx{A} is the matrix of all
zeros.

\item Show that \dltwo{\invs{\Mtx{A}}} equals $\oneBy{\abs{\lambda_{min}}},$
where $\lambda_{min}$ is the smallest, in absolute value, eigenvalue of \Mtx{A}.

\item Suppose there is some $\mu > 0$ such that, for a given \Mtx{A},
\[\dltwo{\Mtx{A}\vect{v}} \ge \mu\dltwo{\vect{v}},\] 
for all vectors \vect{v}.  
	\begin{compactenum}
	\item Show that $\mu \le \dltwo{\Mtx{A}}.$ (Should be very simple.)
	\item Show that \Mtx{A} is nonsingular. (\emph{Recall:} \Mtx{A} is singular if
	there is some $\vect{x} \ne \vect{0}$ such that $\Mtx{A}\vect{x} =
	\vect{0}.$)
	\item Show that $\dltwo{\invs{\Mtx{A}}} \le \oneBy{\mu}.$ 
	\end{compactenum}
\item If \Mtx{A} is singular, is it necessarily the case that
$\dltwo{\Mtx{A}} = 0$?
\item If $\dltwo{\Mtx{A}} \ge \mu > 0$ does it follow that \Mtx{A} is
nonsingular?

\item Towards proving the equality in \bkexpref{maxeig}, prove that
if $\Lambda$ is the set of eigenvalues of $\Mtx{A},$ then
\[
\dabs{\Mtx{A}} \ge \max\limits_{\lambda\in\Lambda} \abs{\lambda},
\]
where \dabs{\cdot} is \emph{any} subordinate matrix norm.
The inequality in the other direction holds when the norm is \dltwo{\cdot}, but
is difficult to prove.
\end{bkexs}
%UNFOLD
%for vim modeline: (do not edit)
% vim:ts=2:sw=2:tw=79:fdm=marker:fmr=FOLDUP,UNFOLD:cms=%%s:tags=tags;:syntax=tex:filetype=tex:ai:si:cin:nu:fo=croqt:cino=p0t0c5(0:
